\anonsection{ЗАКЛЮЧЕНИЕ}

Детектирование рук и распознавание жестов являются одними из
наиболее популярных прикладных задач компьютерного зрения, но
до сих пор в их решении имеются проблемы с получением точного
результата из-за сложности снижения шума. 

Во-первых, для решения задачи детектирования кисти
были рассмотрены различные подходы, а именно детектирование
на основе цветов в разных цветовых моделях (YCbCr и HSV),
разделения пикселей изображения на две группы -- "фон"\ и
"передняя (полезная) часть", и рекурсивные методы
построения модели фона. В результате исследования было получено,
что наиболее эффективным способом в статическом изображении
является детектирование в цветовой модели HSV, а при 
наблюдении в динамике, то есть на видеофрагменте, то 
наилучшие результаты показал метод MoG. 

Во-вторых, для построения контура кисти на изображении были
исследованы два метода: выделение с помощью оператора Кэнни и 
топологический структурный анализ цифрового бинарного
изображения с помощью отслеживания границ. Последний позволяет
находить точные координаты контура, поэтому является наилучшим
среди выбранных.
 
В-третьих, для определения конфигурации кисти, было решено 
использовать её описание в "ключевых"\ точках. Их поиск был
произведён с помощью определения дефектов выпуклости алгоритмами
Грэхема, Джарвиса и Киркпатрика. Сравнение данных алгоритмов
производилось по времени, поскольку именно оно является решающим
в детектировании кисти в реальном времени, при одинаковой точности.
Алгоритм Грэхема показал наилучшие результаты с разницей
более чем в 3 раза по сравнению с алгоритмом Киркпатрика и более
чем в 25 раз по сравнению с алгоритмом Джарвиса. Без дефектов
выпуклости поиск
"ключевых" точек производился с помощью метода локализации на основе
непрерывного скелета, также показавший неплохие результаты.

В-четвёртых, на основе исследований была разработана методика
детектирования и извлечения конфигурации кисти, показавшая
неплохие результаты при несложных движениях кисти человека.

В дальнейшем есть возможность использования полученной конфигурации
кисти для применения в большом множестве задач, включая
распознавание жестов или управление какими-либо автоматическими
системами.
