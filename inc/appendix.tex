\anonsection{ПРИЛОЖЕНИЯ}

\subsection*{Приложение А. Код реализации алгоритма Оцу и построения изображений.}
\addcontentsline{toc}{subsection}{Приложение А}

\begin{minted}[mathescape, 
				linenos,
				gobble=0, 
				frame=lines,
				framesep=2mm]{python}
import numpy as np
import cv2
import matplotlib.pyplot as plt

INPUT_TO_OUTPUT = {'image3.jpg':
                       {'histogram': 'histogram_for_3.png',
                        'output': 'otsu_exmpl_for_3.png'},
                   'image7.jpg':
                       {'histogram': 'histogram_for_7.png',
                        'output': 'otsu_exmpl_for_7.png'}}
def get_min_max(image_flatten: np.ndarray) -> tuple:
    min_pix, max_pix = image_flatten[0], image_flatten[0]
    for pixel in image_flatten:
        if pixel < min_pix:
            min_pix = pixel
        if pixel > max_pix:
            max_pix = pixel
    return min_pix, max_pix
def create_histogram(image_flatten: np.ndarray, 
					size: int, min_pix: int) -> np.array:
    hist = np.zeros(size)
    for pixel in image_flatten:
        hist[pixel - min_pix] += 1
    return hist
def otsu_threshold(image: np.ndarray):
    img_gray = cv2.cvtColor(image, cv2.COLOR_BGR2GRAY)
    img_gray_flatten = img_gray.flatten()
    min_pixel, max_pixel = get_min_max(img_gray_flatten)
    size = max_pixel - min_pixel + 1
    histogram = create_histogram(img_gray_flatten, size, min_pixel)
    m = 0
    n = 0
    for i in range(size):
        m += i * histogram[i]
        n += histogram[i]
    max_sigma = -1
    threshold = 0
    alpha1 = 0
    beta1 = 0
    for i in range(size - 1):
        alpha1 += i * histogram[i]
        beta1 += histogram[i]
        w1 = float(beta1) / n  # Вероятность класса 1
        w2 = 1 - w1
        a = float(alpha1) / beta1 - float(m - alpha1) / (n - beta1) 
        sigma = w1 * w2 * a * a
        if sigma > max_sigma:
            max_sigma = sigma
            threshold = i
    threshold += min_pixel
    return threshold
def save_histogram(image, fname: str):
    image = cv2.cvtColor(image, cv2.COLOR_BGR2GRAY).flatten()
    plt.hist(image, 255)
    plt.xlim([0, 255])
    plt.savefig(fname)
    plt.show()
if __name__ == '__main__':
    for fname, outputs in INPUT_TO_OUTPUT.items():
        image = cv2.imread(fname)
        save_histogram(image, outputs['histogram'])
        thresh = otsu_threshold(image)
        print(f'Threshold = {thresh}')
        ret, img_thresh = cv2.threshold(cv2.cvtColor(image, cv2.COLOR_BGR2GRAY), 
        	thresh, 255, cv2.THRESH_BINARY)
        imask = img_thresh == 255
        canvas = np.zeros_like(image, np.uint8)
        canvas[imask] = image[imask]
        cv2.imwrite(outputs['output'], canvas)
\end{minted}

\newpage

\subsection*{Приложение Б. Код реализации алгоритма Грэхема.}
\addcontentsline{toc}{subsection}{Приложение Б}

\begin{minted}[mathescape, 
				linenos,
				gobble=0, 
				frame=lines,
				framesep=2mm]{python}
from functools import cmp_to_key

class Point:
    def __init__(self, x=None, y=None):
        self.x = x
        self.y = y

# Расстояние между двумя точками
def dist_sq(p1, p2):
    return ((p1.x - p2.x) * (p1.x - p2.x) +
            (p1.y - p2.y) * (p1.y - p2.y))

# Определение ориентации трёх точек
# (перекрестное произведение векторов pq и qr)
def orientation(p, q, r):
    val = ((q.y - p.y) * (r.x - q.x) -
           (q.x - p.x) * (r.y - q.y))
    if val == 0:
        return 0  # коллинеарны
    elif val > 0:
        return 1  # по часовой
    else:
        return 2  # против часовой

# Сравнение двух точек для сортировки
def compare(p1, p2):
    global p0
    o = orientation(p0, p1, p2)
    if o == 0:
        return -1 if dist_sq(p0, p2) >= dist_sq(p0, p1) else 1
    else:
        return -1 if o == 2 else 1

def convex_hull(input_points: list) -> list:
    # Конвертируем в класс Point
    points = [Point(point[0], point[1]) for point in input_points]
    n = len(input_points)
    # Находим минимальную точку
    min_y = points[0].y
    min_i = 0
    for i in range(1, n):
        y = points[i].y
        if ((y < min_y) or
                (min_y == y and points[i].x < points[min_i].x)):
            min_y = points[i].y
            min_i = i

    points[0], points[min_i] = points[min_i], points[0]

    # Сортируем массив
    global p0
    p0 = points[0]
    points = sorted(points, key=cmp_to_key(compare))

    m = 1  # Начальный размер нового массива
    # Удаляем лишние точки
    for i in range(1, n):
        while ((i < n - 1) and
               (orientation(p0, points[i], points[i + 1]) == 0)):
            i += 1
        points[m] = points[i]
        m += 1
    if m < 3:
        return None

    S = [points[0], points[1], points[2]]
    for i in range(3, m):
        while ((len(S) > 1) and
               (orientation(S[-2], S[-1], points[i]) != 2)):
            S.pop()
        S.append(points[i])

    return [(p.x, p.y) for p in S]
\end{minted}

\newpage

\subsection*{Приложение В. Код реализации алгоритма Джарвиса.}
\addcontentsline{toc}{subsection}{Приложение В}

\begin{minted}[mathescape, 
				linenos,
				gobble=0, 
				frame=lines,
				framesep=2mm]{python}
class Point:
    def __init__(self, x, y):
        self.x = x
        self.y = y

# поиск самой левой нижней точки
def left_index(points):
    minn = 0
    for i in range(1, len(points)):
        if points[i].x < points[minn].x or \
                (points[i].x == points[minn].x and
                 points[i].y > points[minn].y):
            minn = i
    return minn

# аналогично orientation в алгоритме Грэхема
def orientation(p, q, r):
    val = (q.y - p.y) * (r.x - q.x) - \
          (q.x - p.x) * (r.y - q.y)
    if val == 0:
        return 0
    elif val > 0:
        return 1
    return -1

def convex_hull(init_points):
    n = len(init_points)
    if n < 3:
        return
    points = [Point(p[0], p[1]) for p in init_points]
    l = left_index(points)
    hull = []
    p = l
    q = 0
    while True:
        hull.append(p)
        q = (p + 1) % n
        for i in range(n):
            if (orientation(points[p],
                            points[i], points[q]) == 2):
                q = i
        p = q
        if p == l:
            break
    return [(points[i].x, points[i].y) for i in hull]
\end{minted}

\newpage

\subsection*{Приложение Г. Код реализации алгоритма Киркпатрика.}
\addcontentsline{toc}{subsection}{Приложение Г}

\begin{minted}[mathescape, 
				linenos,
				gobble=0, 
				frame=lines,
				framesep=2mm]{python}
from collections import namedtuple
from random import randint

Point = namedtuple('Point', 'x y')

def flipped(points):
    return [Point(-point.x, -point.y) for point in points]

def quickselect(ls, index, lo=0, hi=None, depth=0):
    if hi is None:
        hi = len(ls) - 1
    if lo == hi:
        return ls[lo]
    if len(ls) == 0:
        return 0
    pivot = randint(lo, hi)
    ls = list(ls)
    ls[lo], ls[pivot] = ls[pivot], ls[lo]
    cur = lo
    for run in range(lo+1, hi+1):
        if ls[run] < ls[lo]:
            cur += 1
            ls[cur], ls[run] = ls[run], ls[cur]
    ls[cur], ls[lo] = ls[lo], ls[cur]
    if index < cur:
        return quickselect(ls, index, lo, cur-1, depth+1)
    elif index > cur:
        return quickselect(ls, index, cur+1, hi, depth+1)
    else:
        return ls[cur]

def bridge(points, vertical_line):
    candidates = set()
    if len(points) == 2:
        return tuple(sorted(points))
    pairs = []
    modify_s = set(points)
    while len(modify_s) >= 2:
        pairs += [tuple(sorted([modify_s.pop(), modify_s.pop()]))]
    if len(modify_s) == 1:
        candidates.add(modify_s.pop())
    slopes = []
    for pi, pj in pairs[:]:
        if pi.x == pj.x:
            pairs.remove((pi, pj))
            candidates.add(pi if pi.y > pj.y else pj)
        else:
            slopes += [(pi.y-pj.y)/(pi.x-pj.x)]
    median_index = len(slopes)//2 - (1 if len(slopes) % 2 == 0 else 0)
    median_slope = quickselect(slopes, median_index)
    small = {pairs[i] for i, slope in enumerate(slopes) if slope < median_slope}
    equal = {pairs[i] for i, slope in enumerate(slopes) if slope == median_slope}
    large = {pairs[i] for i, slope in enumerate(slopes) if slope > median_slope}
    max_slope = max(point.y-median_slope*point.x for point in points)
    max_set = [point for point in points if \
    	point.y-median_slope*point.x == max_slope]
    left = min(max_set)
    right = max(max_set)
    if left.x <= vertical_line < right.x:
        return left, right
    if right.x <= vertical_line:
        candidates |= {point for _, point in large | equal}
        candidates |= {point for pair in small for point in pair}
    if left.x > vertical_line:
        candidates |= {point for point, _ in small | equal}
        candidates |= {point for pair in large for point in pair}
    return bridge(candidates, vertical_line)

def connect(lower, upper, points):
    if lower == upper:
        return [lower]
    max_left = quickselect(points, len(points)//2-1)
    min_right = quickselect(points, len(points)//2)
    left, right = bridge(points, (max_left.x + min_right.x)/2)
    points_left = {left} | {point for point in points if point.x < left.x}
    points_right = {right} | {point for point in points if point.x > right.x}
    return connect(lower, left, points_left) + connect(right, upper, points_right)

def upper_hull(points):
    lower = min(points)
    lower = max({point for point in points if point.x == lower.x})
    upper = max(points)
    points = {lower, upper} | {p for p in points if lower.x < p.x < upper.x}
    return connect(lower, upper, points)

def convex_hull(init_points):
    points = [Point(p[0], p[1]) for p in init_points]
    upper = upper_hull(points)
    lower = flipped(upper_hull(flipped(points)))
    if upper[-1] == lower[0]:
        del upper[-1]
    if upper[0] == lower[-1]:
        del lower[-1]
    return upper + lower
\end{minted}

\newpage

\subsection*{Приложение Д. Код функции очистки лишних точек дефектов выпуклости.}
\addcontentsline{toc}{subsection}{Приложение Д}

\begin{minted}[mathescape, 
				linenos,
				gobble=0, 
				frame=lines,
				framesep=2mm]{python}
import math
import numpy as np

def squeeze(array):
    array = array[0] if len(array) == 1 else array
    return [p[0] for p in array] if len(array[0]) == 1 else array

# "очистка" выпуклой оболочки до не больше 7 точек
def clear_convex_hull(hull_index, contour):
    distance_point = lambda p1, p2: \
    	math.sqrt((p1[0] - p2[0]) ** 2 + (p1[1] - p2[1]) ** 2)
    hull_index, contour = squeeze(hull_index), squeeze(contour)
    ALPHA = 10
    while True:
        boundary = len(hull_index) - 1
        clean_hull, i = [], 0
        while i < boundary:
            clean_hull.append(hull_index[i])
            while i < boundary and \
                    distance_point(contour[hull_index[i]],
                    contour[hull_index[i + 1]]) < ALPHA:
                i += 1
            i += 1
        if len(clean_hull) > 7:
            ALPHA += 1
        else:
            break
    return np.array(clean_hull[:-1]) if \
        distance_point(contour[clean_hull[0]],
        contour[clean_hull[-1]]) < ALPHA else np.array(clean_hull)
\end{minted}
