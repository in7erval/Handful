\section{МЕТОДЫ ПОСТРОЕНИЯ КОНТУРА ОБЪЕКТА НА ИЗОБРАЖЕНИИ}

Отслеживание границ -- один из основных методов обработки оцифрованных
двоичных изображений. Он производит последовательность координат или 
цепных кодов от границ между связным компонентом.

\subsection{Выделение границ с помощью оператора Кэнни}

Оператор Кэнни (детектор границ Кэнни или алгоритм Кэнни) -- оператор
обнаружения границ изображения. Кэнни изучил математическую проблему
получения фильтра, оптимального по критериям выделения, локализации и
минимизации нескольких откликов одного края. Он показал, что искомый
фильтр является суммой четырёх экспонент. Он также показал, что этот
фильтр может быть хорошо приближен первой производной Гауссианы. Кэнни
ввёл понятие {\it подавление немаксимумов} (Non-Maximum Suppression), 
которое означает, что пикселями границ объявляются пиксели, в которых
достигается локальный максимум градиента в направлении вектора
градиента.

Целью Кэнни было разработать оптимальный алгоритм обнаружения границ, 
удовлетворяющий трём критериям:
\begin{itemize}
	\item хорошее обнаружение (повышение отношения сигнал/шум);
	\item хорошая локализация (правильное определения положения
границы);
	\item единственный отклик на одну границу.
\end{itemize}

Из этих критериев затем строится целевая функция стоимости ошибок, 
минимизацией которой находится "оптимальный"\ линейный оператор для
свёртки с изображением.

Первым этапом алгоритма является {\it сглаживание}, то есть размытие
изображения для удаления шума. Оператор Кэнни использует фильтр, 
который может быть хорошо приближен к первой производной гауссианы.
Для $\sigma=1.4$:

\begin{equation}
	B = \frac{1}{159}
\begin{bmatrix}
	2 & 4 & 5 & 4 & 2\\
	4 & 9 & 12 & 9 & 4\\
	5 & 12 & 15 & 12 & 5\\
	4 & 9 & 12 & 9 & 4\\
	2 & 4 & 5 & 4 & 2
\end{bmatrix}
\cdot A.
\label{canny-1}
\end{equation}

Следующим этапом является {\it поиск градиентов}. Границы отмечаются 
там, где градиент изображения приобретает максимальное значение. Они 
могут иметь различное направление, поэтому алгоритм Кэнни использует
четыре фильтра для обнаружения горизонтальных, вертикальных и 
диагональных ребер в размытом изображении.
\begin{equation}
    G = \sqrt{G_x^2 + G_y^2},
    \Theta = \arctan (\frac{G_y}{G_x}).
	\label{canny-2}
\end{equation}

Угол направления вектора градиента при этом округляется и может 
принимать такие значения: 0\textdegree, 45\textdegree, 90\textdegree,
135\textdegree.

Затем происходит {\it подавление немаксимумов}, когда только локальные
максимумы отмечаются как границы, {\it двойная пороговая филтьтрация}
-- потенциальные границы определяются порогами и {\it трассировка
области неоднозначности}, когда итоговые границы определяются путём 
подавления всех краёв, не связанных с определёнными (сильными) 
границами.

Перед применением детектора обычно преобразуют изображение в оттенки
серого, чтобы уменьшить вычислительные затраты. 

Пример работы оператора Кэнни в цветовых моделях RBG, GRAY, HSV и YCbCr 
показан на рис. \ref{canny-img1}.



Можно видеть, что алгоритм Кэнни довольно неплохо позволяет определить
границы объекта. Но для задачи извлечения конфигурации кисти 
необходимо извлекать особые точки изображения, а этот метод лишь 
выделяет границы на изображении, не определяя сам контур. В связи с
этим необходимо рассмотреть топологический структурный анализ цифрового
бинарного изображения с помощью отслеживания границ.

\subsection{Топологический структурный анализ цифрового бинарного
изображения с помощью отслеживания границ}

Этот метод был разработан Сатоши Сузуки и Кейчи Эйбом в 1985 году
\cite{satoshi}. Алгоритм предполагает нахождение контуров с учетом
вложенности, то есть способен определить, когда в контур одного объекта
вложен другой. Реализация данного алгоритма лежит в основе функции
{\tt findContours()} в библиотеке OpenCV, предназначенной для 
исследования и решения задач, связанных с компьютерным зрением. 

Для отрисовки контуров, полученных с помощью данной функции, можно воспользоваться
функцией {\tt drawContours()}.

Поскольку функция {\tt findContours()} находит все контуры на 
изображении, то нужно среди них выбрать один единственный. 
Пусть кисть занимает наибольшее пространство на изображении и, 
соответственно, имеет наибольший контур. Среди всех контуров будем 
отбирать тот, {\it площадь} которого имеет наибольшее значение. 
Площадь контура будем находить с помощью {\it формулы площади Гаусса}.

\addtwoimgherepro{pix/image1}{pix/canny_image1}{Пример работы
оператора Кэнни.}{canny-img1}{0.3}{0.8}

\subsubsection{Формула площади Гаусса}

Формула площади Гаусса (формула землемера или формула шнурования или
алгоритм шнурования) — формула определения площади простого
многоугольника, вершины которого заданы декартовыми координатами на
плоскости. В формуле векторным произведением координат и сложением
определяется площадь области, охватывающей многоугольник, а затем из
нее вычитается площадь окружающего многоугольника, что дает площадь
многоугольника внутри. 

Формула может быть представлена следующим выражением:
\begin{equation}
\begin{aligned}
	&S = \frac{1}{2} \left| 
	\sum_{i=1}^{n-1}{x_i y_{i+1}}+x_n y_1 -
	\sum_{i=1}^{n-1}{x_{i+1} y_i} - x_1 y_n
	\right|=\\
	&=\frac{1}{2} \left| x_1 y_2 + x_2 y_3 + 
	\dots + x_{n-1} y_n + x_n y_1 - x_2 y_1
	-x_3 y_2 - \dots - x_n y_{n-1} - x_1 y_n\right|,
\end{aligned}
\label{gauss-square-equation}
\end{equation}
где

$S$ -- площадь многоугольника,

$n$ -- количество сторон многоугольника,

$(x_i, y_i), i=\overline{1,n}$ -- координаты вершин многоугольника.

Другие представления этой же формулы:
\begin{equation}
\begin{aligned}
	&S = \frac{1}{2} \left| 
	\sum_{i=1}^n{x_i (y_{i+1}-y_{i-1})} \right| =
	\frac{1}{2} \left| 
	\sum_{i=1}^n{y_i (x_{i+1}-x_{i-1})} \right| =\\
	&= \frac{1}{2} \left| 
	\sum_{i=1}^n{x_i y_{i+1} - x_{i+1} y_i} \right| =
	\frac{1}{2} \left| 
	\sum_{i=1}^n{det
	\begin{pmatrix}
		x_i & y_i \\
		x_{i+1} & y_{i+1}
	\end{pmatrix}
	} \right|,
\end{aligned}
\label{gauss-square-equation-1}
\end{equation}
где

$x_{n+1}=x_1,~x_0=x_n$,

$y_{n+1}=y_1,~y_0=y_n$.

\bigskip

Рассмотрим пример использования данных функций. Сравним результат
поиска контуров с предварительной обработкой изображения, 
заключающейся в отделении объекта от фона, и без обработки (только с 
переводом изображения в черно-белый формат) и 
получим результаты, представленные на рис. \ref{contours-img-ex}.

\addtwoimghere{pix/contours_otsu_example}
{pix/contours_otsu_processing_example}
{Пример поиска контура на изображении без предварительной обработки
(а) и с предварительной обработкой с помощью метода Оцу (б).}
{contours-img-ex}

Можно видеть, что пороговая бинаризация изображения методом Оцу 
позволила с высокой точностью определить истинное расположение контура
ладони. Это означает, что для дальнейшего исследования контуров
необходима предобработка изображения.


