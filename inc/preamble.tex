%%%% Преамбула %%%

\usepackage[usenames,dvipsnames]{color}
\usepackage{geometry}
\geometry{left=3cm,right=1cm,top=2cm,bottom=2cm}
\usepackage[utf8]{inputenc}
\usepackage{mathtext}
\usepackage[T2A]{fontenc}
%\inputencoding{cp1251}          % тоже кодировка...
\usepackage[russian]{babel}
\usepackage[unicode]{hyperref}
\usepackage{amstext,amsmath,amssymb}
\usepackage{bm}
\usepackage[pdftex]{graphicx}
\usepackage{titlesec}
\usepackage[title,titletoc]{appendix}
\usepackage{titletoc}

\usepackage{amsfonts}           % греческие символы и, возможно, что-то ещё
\usepackage{indentfirst}        % одинаковый отступ для первого параграфа и всего остального
\usepackage{cite}               % команда /cite{1,2,7,9} даёт ссылки
\usepackage{multirow}           % пакет для объединения строк в таблице: надо указать число строк и ширину столбца
\usepackage{array}              % нужен для создания таблиц
\linespread{1.3}                % полтора интервала. Если 1.6, то два интервала
\pagestyle{plain}               % номерует страницы
\usepackage{misccorr}
\usepackage{listings} % Оформление исходного кода
\lstset{
	language=Python,
    basicstyle=\small\ttfamily, % Размер и тип шрифта
    breaklines=true, % Перенос строк
    tabsize=2, % Размер табуляции
    literate={--}{{-{}-}}2 % Корректно отображать двойной дефис
}
\usepackage{pdfpages} % Вставка PDF
\usepackage{hyphenat}

%
%% Шрифты, xelatex

%
%% Русский язык
%\usepackage{polyglossia}
%\usepackage{amssymb,amsfonts,amsmath} % Математика
%\numberwithin{equation}{section} % Формула вида секция.номер
%
%\usepackage{enumerate} % Тонкая настройка списков
%\usepackage{indentfirst} % Красная строка после заголовка
%\usepackage{float} % Расширенное управление плавающими объектами
%\usepackage{multirow} % Сложные таблицы
%
%% Формат подрисуночных записей
%\usepackage{chngcntr}
%\counterwithin{figure}{section}
%
%% Гиперссылки
\usepackage{hyperref}
\hypersetup{
    colorlinks, urlcolor={black}, % Все ссылки черного цвета, кликабельные
    linkcolor={black}, citecolor={black}, filecolor={black},
    pdfauthor={Дмитрий Юдаков},
    pdftitle={Исследование методов и разработка алгоритма детектирования, описания и извлечения конфигураций кисти человека}
}
%
%% Оформление библиографии и подрисуночных записей через точку
%\makeatletter
%\renewcommand*{\@biblabel}[1]{\hfill#1.}
%\renewcommand*\l@section{\@dottedtocline{1}{1em}{1em}}
%\renewcommand{\thefigure}{\thesection.\arabic{figure}} % Формат рисунка секция.номер
%\renewcommand{\thetable}{\thesection.\arabic{table}} % Формат таблицы секция.номер
%\def\redeflsection{\def\l@section{\@dottedtocline{1}{0em}{10em}}}
%\makeatother
%
\renewcommand{\baselinestretch}{1.4} % Полуторный межстрочный интервал
\parindent 1.25cm % Абзацный отступ
%
%\sloppy             % Избавляемся от переполнений
%\hyphenpenalty=1000 % Частота переносов
\clubpenalty=10000  % Запрещаем разрыв страницы после первой строки абзаца
\widowpenalty=10000 % Запрещаем разрыв страницы после последней строки абзаца
%

%% Списки
\usepackage{enumitem}
\setlist[enumerate,itemize]{leftmargin=12.5mm} % Отступы в списках
%
%\makeatletter
%    \AddEnumerateCounter{\asbuk}{\@asbuk}{м)}
%\makeatother
\renewcommand{\theenumi}{\arabic{enumi}}

\setlist{nolistsep} % Нет отступов между пунктами списка
%\renewcommand{\labelitemi}{--} % Маркет списка --
\renewcommand{\labelenumi}{\arabic{enumi}. } % Список второго уровня
\renewcommand{\labelenumii}{\arabic{enumi}.\arabic{enumii}.}
%% Содержание
\usepackage{tocloft}
\renewcommand{\cftsecfont}{\hspace{0pt}} % Имена секций в содержании не жирным шрифтом
\renewcommand\cftsecleader{\cftdotfill{\cftdotsep}} % Точки для секций в содержании
\renewcommand\cftsecpagefont{\mdseries} % Номера страниц не жирные
\setcounter{tocdepth}{3} % Глубина оглавления, до subsubsection
\renewcommand{\cftsecaftersnum}{.}
%
%% Формат подрисуночных надписей
\RequirePackage{caption}
\DeclareCaptionLabelSeparator{defffis}{. } % Разделитель
\captionsetup[figure]{justification=centering, labelsep=defffis, format=plain} % Подпись рисунка по центру
\captionsetup[table]{justification=centering, labelsep=defffis, format=plain, singlelinecheck=false} % Подпись таблицы слева
\addto\captionsrussian{\renewcommand{\figurename}{Рис.}} % Имя фигуры

\newcommand{\addimg}[4]{ % Добавление одного рисунка
    \begin{figure}
        \centering
        \includegraphics[width=#2\linewidth]{#1}
        \caption{#3} \label{#4}
    \end{figure}
}

\newcommand{\addimghere}[4]{ % Добавить рисунок непосредственно в это место
    \begin{figure}[H]
        \centering
        \includegraphics[width=#2\linewidth]{#1}
        \caption{#3} \label{#4}
    \end{figure}
}

\newcommand{\addtwoimghere}[4]{
	\begin{figure}[h]
		\begin{minipage}[h]{0.5\linewidth}
			\center{\includegraphics[width=0.9\linewidth]{#1} \\ а)}
		\end{minipage}
		\hfill
		\begin{minipage}[h]{0.5\linewidth}
			\center{\includegraphics[width=0.9\linewidth]{#2} \\ б)}
		\end{minipage}
		\caption{#3}
		\label{#4}
	\end{figure}
}

\newcommand{\addtwoimgherepro}[6]{
	\begin{figure}[h]
		\begin{minipage}[h]{#5\linewidth}
			\center{\includegraphics[width=1\linewidth]{#1} \\ а)}
		\end{minipage}
		\hfill
		\begin{minipage}[h]{#6\linewidth}
			\center{\includegraphics[width=1\linewidth]{#2} \\ б)}
		\end{minipage}
		\caption{#3}
		\label{#4}
	\end{figure}
}

\newcommand{\addimgsandhistshere}[6]{
	\begin{figure}[hp]
		\begin{minipage}[hp]{0.3\linewidth}
			\center{\includegraphics[width=0.9\linewidth]{#1}} \\a) 
		\end{minipage}
		\begin{minipage}[hp]{0.7\linewidth}
			\center{\includegraphics[width=0.9\linewidth]{#2}} \\б)
		\end{minipage}
		\vfill
		\begin{minipage}[hp]{0.3\linewidth}
			\center{\includegraphics[width=0.9\linewidth]{#3}} \\в)
		\end{minipage}
		\begin{minipage}[hp]{0.7\linewidth}
			\center{\includegraphics[width=0.9\linewidth]{#4}} \\г)
		\end{minipage}
		\caption{#5}
		\label{#6}
	\end{figure}
}

\newcommand{\addtwoxtwoimghere}[6]{
	\begin{figure}[hp]
		\begin{minipage}[hp]{0.5\linewidth}
			\center{\includegraphics[width=0.7\linewidth]{#1}} \\a) 
		\end{minipage}
		\begin{minipage}[hp]{0.5\linewidth}
			\center{\includegraphics[width=0.7\linewidth]{#2}} \\б)
		\end{minipage}
		\vfill
		\begin{minipage}[hp]{0.5\linewidth}
			\center{\includegraphics[width=0.7\linewidth]{#3}} \\в)
		\end{minipage}
		\begin{minipage}[hp]{0.5\linewidth}
			\center{\includegraphics[width=0.7\linewidth]{#4}} \\г)
		\end{minipage}
		\caption{#5}
		\label{#6}
	\end{figure}
}

\addto\captionsrussian{
\def\refname{\begin{center}СПИСОК ИСПОЛЬЗОВАННОЙ ЛИТЕРАТУРЫ \end{center}}
\def\contentsname{\begin{center} СОДЕРЖАНИЕ \end{center}}
}

%
%% Заголовки секций в оглавлении в верхнем регистре
%\usepackage{textcase}
%\makeatletter
%\let\oldcontentsline\contentsline
%\def\contentsline#1#2{
%    \expandafter\ifx\csname l@#1\endcsname\l@section
%        \expandafter\@firstoftwo
%    \else
%        \expandafter\@secondoftwo
%    \fi
%    {\oldcontentsline{#1}{\MakeTextUppercase{#2}}}
%    {\oldcontentsline{#1}{#2}}
%}
%\makeatother
%
%% Оформление заголовков
%\usepackage[compact,explicit]{titlesec}
%\titleformat{\section}{}{}{12.5mm}{\centering{\thesection\quad\MakeTextUppercase{#1}}\vspace{1.5em}}
%\titleformat{\subsection}[block]{\vspace{1em}}{}{12.5mm}{\thesubsection\quad#1\vspace{1em}}
%\titleformat{\subsubsection}[block]{\vspace{1em}\normalsize}{}{12.5mm}{\thesubsubsection\quad#1\vspace{1em}}
%\titleformat{\paragraph}[block]{\normalsize}{}{12.5mm}{\MakeTextUppercase{#1}}
%
%% Секции без номеров (введение, заключение...), вместо section*{}
\newcommand{\anonsection}[1]{
\section*{\centering #1}
\addcontentsline{toc}{section}{#1}
}

%
%% Секции для приложений
%\newcommand{\appsection}[1]{
%    \phantomsection
%    \paragraph{\centerline{{#1}}}
%    \addcontentsline{toc}{section}{\uppercase{#1}}
%}
%
%% Библиография: отступы и межстрочный интервал
\makeatletter
\renewenvironment{tableofcontents}[1]
    {\section*{\refname}
        \list{\@biblabel{\@arabic\c@enumiv}}
           {\settowidth\labelwidth{\@biblabel{#1}}
            \leftmargin\labelsep
            \itemindent 16.7mm
            \@openbib@code
            \usecounter{enumiv}
            \let\p@enumiv\@empty
            \renewcommand\theenumiv{\@arabic\c@enumiv}
        }
        \setlength{\itemsep}{0pt}
    }
\makeatother

\usepackage{listings}
\usepackage{color}
\usepackage{minted}
\usepackage{python}
%\definecolor{dkgreen}{rgb}{0,0.6,0}
%\definecolor{gray}{rgb}{0.5,0.5,0.5}
%\definecolor{mauve}{rgb}{0.58,0,0.82}
%
\lstset{ %
language=Python,                 % выбор языка для подсветки (здесь это С)
basicstyle=\small\sffamily, % размер и начертание шрифта для подсветки кода
numbers=left,               % где поставить нумерацию строк (слева\справа)
numberstyle=\tiny,           % размер шрифта для номеров строк
stepnumber=1,                   % размер шага между двумя номерами строк
numbersep=5pt,                % как далеко отстоят номера строк от подсвечиваемого кода
backgroundcolor=\color{white}, % цвет фона подсветки - используем \usepackage{color}
showspaces=false,            % показывать или нет пробелы специальными отступами
showstringspaces=false,      % показывать или нет пробелы в строках
showtabs=false,             % показывать или нет табуляцию в строках
frame=single,              % рисовать рамку вокруг кода
tabsize=2,                 % размер табуляции по умолчанию равен 2 пробелам
captionpos=t,              % позиция заголовка вверху [t] или внизу [b] 
breaklines=true,           % автоматически переносить строки (да\нет)
breakatwhitespace=false, % переносить строки только если есть пробел
escapeinside={\%*}{*)}   % если нужно добавить комментарии в коде
}


  \makeatletter
  \renewcommand*{\@biblabel}[1]{\hfill#1.}
  \makeatother


    
\sloppy

%\setcounter{page}{4} % Начало нумерации страниц