\section{\nohyphens{МЕТОДЫ ДЕТЕКТИРОВАНИЯ КИСТИ ЧЕЛОВЕКА В СИСТЕМАХ ЧЕЛОВЕКО-МАШИННОГО ВЗАИМОДЕЙСТВИЯ}}

Человеко-машинное взаимодействие (Human-computer interaction - HCI) - это междисциплинарное
научное направление, изучающее взаимодействие между людьми и машинами. Предметом HCI является
изучения, планирование и разработка методов взаимодействия человека с машиной, где в роли машины
может выступать персональный компьютер, компьютерная система больших масштабов, система 
управления процессами и т.д. \cite{dix}. Под взаимодействием понимается любая коммуникация между
человеком и машиной. Одним из методов HCI, получившим широкое распространение в последние годы,
является взаимодействие, основанное на жестах человека \cite{jiangqin, sanna}. 

Задачу распознавания жестов руки можно разделить на подзадачи:
\begin{enumerate}
	\item Отделение кисти руки от остальной части изображения.
	\item Построение контура кисти.
	\item Нахождение ключевых точек на кисти.
	\item Классификация жеста исходя из статического или динамического расположения точек.
\end{enumerate}

Первая подзадача, а именно детектирование кисти человека в кадре является ключевой, поскольку
от качества её решения зависит качество выполнения остальных подзадач.
Рассмотрим первую подзадачу, а именно детектирование кисти человека в кадре.

Существует множество решений этой подзадачи. Наиболее популярными из них являются 
отделение фона изображения, распознавание цвета 
кожи в кадре и метод Оцу. 
Подробно рассмотрим каждое из них.

\subsection{Отделение фона изображения}

В данном решении принимается, что в кадре движется только рука, а остальные части
тела, включая фон, остаются неподвижными. Таким образом, если вначале инициализировать фон как
$bgr(x, y)$, а новое изображение с жестом рассматривать как $fgr(x, y)$, то изолированный жест
можно принять как разность между этими изображениями: 
\begin{equation} gst_i(x,y)=fgr_i(x,y)-bgr(x,y). \label{first}\end{equation}

Полученный разностный жест переднего плана преобразуется в бинарное изображение, устанавливая
соответствующий порог. Поскольку фон не является полностью статичным, например, если камера
удерживается в руках оператора, то добавляется шумовая часть. Чтобы получить изображение руки
без шумов, этот метод сочетается с распознаванием кожи человека. Чтобы удалить этот шум, 
применяется анализ связанных компонентов, чтобы заполнить пустоты, если применяется заливка
какой-либо области, а также для получения чётких краёв применяется морфологическая обработка.

Недостаток данного решения состоит в том, что довольно сложно отделить фон, даже если он задан,
поскольку не ясно какие из пикселей изменились, а какие остались прежними из-за тени, смещения
фокуса, изменения экспозиции и т.д. Даже если зафиксировать все параметры камеры, то тень так
или иначе испортит качество решения.

Приведём два примера отделения кисти от фона изображения. На рис. \ref{pix1} изображены два
изображения, одно из которых является инициализированным фоном, а второе -- кисть на этом фоне.

\addtwoimghere{pix/BG_result}{pix/FG_result}{Инициализированный фон (а) и изображение
кисти на нём (б).}{pix1}
Для отделения кисти первым способом воспользуемся встроенной в библиотеку OpenCV функцией
{\tt absdiff()}.

Второй способ будет заключаться в следующем. Пусть даны два пикселя 
\begin{equation}
	p_1 = (r_1, g_1, b_1)~~и~~p_2 = (r_2, g_2, b_2),
	\label{pixels_bg}
\end{equation}
тогда различие между ними будем определять как
\begin{equation}
	D = \sqrt{(r_2 - r_1)^2 + (g_2 - g_1)^2 + (b_2 - b_1)^2}.
	\label{distance}
\end{equation}
Затем создаём битовую маску $M$, в которой значения определяются как
\begin{equation}
	M_{(i,j)}=\left\{\begin{aligned}
	1,~& D > Threshold\\
	0,~& D \leq Threshold
\end{aligned}\right. 
\label{bite-mask}
\end{equation}

Результаты работы первого и второго метода представлены на рис. \ref{pix2}.
\addtwoimghere{pix/result}{pix/result1}{Результат первого метода (а) и второго (б).}{pix2}

Как можно видеть, оба метода не совсем точно отделяют нужный объект от фона, поэтому 
рассмотрим метод распознавания кожи в HSV и YCbCr цветовых моделях.

\newpage

\subsection{Метод распознавания кожи в HSV и YCbCr цветовых моделях}

Для того, чтобы детектировать цвет кожи на изображениях очень часто применяются 
различные цветовые модели, а именно HSV и YCbCr.

{\bf HSV} ({\it Hue, Saturation, Value}) или {\bf HSB} ({\it Hue, Saturation, Brightness}) --
цветовая модель, в которой координатами цвета являются:
\begin{enumerate}
	\item {\bf H}ue -- цветовой тон, (например, красный, зелёный или сине-голубой). Варьируется 
в пределах 0-360\textdegree, однако иногда приводится к диапазону 0-100 или 0-1.
	\item {\bf S}aturation -- насыщенность. Варьируется в пределах 0-100 или 0-1. Чем больше
этот параметр, тем "чище"\ цвет, поэтому иногда называют {\it чистотой цвета}. А чем ближе этот
параметр к нулю, тем ближе цвет к нейтральному серому.
	\item {\bf V}alue или {\bf B}rightness -- яркость. Также задаётся в пределах 0-100 или 0-1.
\end{enumerate}

{\bf YCbCr} -- семейство цветовых пространств, которые используются для передачи цветных
изображений в компонентном видео и цифровой фотографии. Является частью рекомендации 
МСЭ-R ВT.601 при разработке стандарта видео Всемирной цифровой организации и фактически
является масштабированной и смещённой копией YUV. 

{\bf Y} -- компонента яркости, {\bf Cb} -- синий компонент цветности, 
{\bf Cr} -- красный компонент цветности.

Пример изображения в HSV изображён на рисунке \ref{rgb-hsv-pic}.
\addtwoimgherepro{pix/FG_result}{pix/hsv_example}{Изображение в RGB (а) и HSV (б).}
{rgb-hsv-pic}{0.39}{0.39}

Пример изображения в $YC_BC_R$ изображен на рисунке \ref{rgb-ycbcr-pic}.
\addtwoimgherepro{pix/FG_result}{pix/ycbcr_example}{Изображение в RGB (а) и $YC_BC_R$ (б).}
{rgb-ycbcr-pic}{0.39}{0.39}

Для того чтобы отделить кожу от остальной части изображения, 
используют определённые диапазоны составляющих цветовых моделей, 
которые находятся эмпирически или итеративно. В первом случае путём 
проб и ошибок подбираются значения составляющих. Можно заметить,
что при данном подходе кожа будет иметь разные цветовые диапазоны
при разном освещении. Покажем это. Зададим диапазон для 
изображения в HSV цветовой модели как

$$
\begin{aligned}
	&0 \leq H \leq 200, \\
	&15 \leq S \leq 255, \\
	&80 \leq V \leq 255,
\end{aligned}
$$
и получим результат, представленный на рис. \ref{hsv-del-pic}.

Можно видеть, что на первом изображении фон отделился практически 
идеально, но на втором видны фрагменты фона.

\newpage

\addtwoxtwoimghere{pix/image3}{pix/hsv_del1}{pix/image7}{pix/hsv_del2}{Изображения с удалённым
фоном в HSV.}{hsv-del-pic}

Для цветовой модели YCbCr в статье \cite{ycbcr-bib} предложили два 
варианта диапазона компонент (\ref{ycbcr-diap1}) и (\ref{ycbcr-diap2}):

\begin{equation}
	\begin{aligned}
		&80 < Y \leq 255, \\
		&85 < C_b < 135, \\
		&135 < C_r < 180 
	\end{aligned}
	\label{ycbcr-diap1}
\end{equation}

\begin{equation}
	\begin{aligned}
		&Y \in \forall,\\
		&77 \leq C_b \leq 127, \\
		&133 \leq C_r \leq 173 
	\end{aligned}
	\label{ycbcr-diap2}
\end{equation}

Сравнение двух вариантов представлено на рисунке \ref{ycbcr-del-pic}.

\addtwoxtwoimghere{pix/ycbcr_del1_1}{pix/ycbcr_del1_2}
{pix/ycbcr_del2_1}{pix/ycbcr_del2_2}{Изображения с удалённым
фоном в YCbCr.}{ycbcr-del-pic}

Легко видеть, что оба диапазона работают не идеально, но первый 
показал себя намного лучше. 

Таким образом, можно сделать вывод, что для детектирования цвета кожи 
лучше брать изображение в цветовой модели HSV. 

Рассмотрим следующий метод отделения фона изображения, а именно метод
Оцу.

\subsection{Метод Оцу}

В 1979 году Нобуюки Оцу опубликовал статью \cite{otsu} метода порогового 
разделения, основываясь на гистограмме серых цветов изображения. 

Метод Оцу -- это алгоритм вычисления порога бинаризации для полутонового
изображения, используемый в области компьютерного распознавания образов
и обработки изображений для получения чёрно-белых изображений. Алгоритм
позволяет разделить пиксели двух классов ("полезные"\ и "фоновые"), 
рассчитывая такой порог, чтобы внутриклассовая дисперсия была
минимальной. Для того чтобы упростить алгоритм, используется
гистограмма монохромного изображения.
 
Диаграммы для изображений изображены на рис. \ref{histograms}. 
Как видно из рисунков, на данных изображениях довольно
проблематично отделить фон от изображения какой-то единственной 
пороговой величиной, поэтому и алгоритм Оцу на таких изображениях 
сработает не очень хорошо, как это показано на рис. \ref{otsu-ex}. 
Реализация данного алгоритма и построение гистограмм изображений
находятся в приложении А.

\addimgsandhistshere{pix/image3}{pix/histogram_for_3}
{pix/image7}{pix/histogram_for_7}{Изображение 1 и 2
и их гистограммы.}{histograms}

\newpage

\addtwoimghere{pix/otsu_exmpl_for_3}{pix/otsu_exmpl_for_7}{Результат работы программы с алгоритмом Оцу}{otsu-ex}

\subsection{Mixture of Gaussians}

Стандартным подходом к построению модели фона, использующимся для многих
прикладных задач, является смесь гауссовых распределений (Mixture of Gaussians,
MOG) \cite{MOG-1, MOG-2}. Чаще всего для каждого пикселя
текущего кадра с номером $n$ строится функция плотности вероятности 
$P_n=P(\nu_n(p))$, и MOG используется именно этот подход. Предполагается, что 
для каждого пикселя текущего изображения она может быть представлена смесью
нормальных распределений, где $G$ -- их число в смеси.

Для инициализации гауссиан для каждого пикселя чаще всего применяют либо
EM-алгоритм (Expectation-maximization algorithm), либо k-means, что
достаточно затратно в вычислительном плане. Число входящих в смесь распределений
$G$ обычно принимают равным от 3 до 5. Также существует подход, позволяющий
автоматически подбирать необходимое количество гауссиан \cite{MOG-2}. 

В библиотеке OpenCV данный метод реализуется с помощью встроенной функции
{\tt createBackgroundSubtractorMOG2()}. Данная функция обучается на изображениях
каждого кадра и лучше всего работает в режиме последовательности, то есть на
видео. Результат работы метода MoG представлен на рисунке
\ref{mog-example-img}.

\newpage

\addimghere{pix/mog_result}{1}
{Результат отделения фона от изображения с помощью метода MoG.}{mog-example-img}

В коде видно, что сначала происходит обучение на первых изображениях фона, 
а затем добавляется новый кадр, на котором стирается фон.

\bigskip

Рассмотрев наиболее популярные методы отделения кисти человека от фона изображения, можно сделать
вывод, что некоторые из методов работают наилучшим образом при специфических условиях, поэтому
необходимо их дополнительное исследование. 









