\newpage
\begin{thebibliography}{99}
\addcontentsline{toc}{section}{СПИСОК ИСПОЛЬЗОВАННОЙ ЛИТЕРАТУРЫ}

\bibitem{mesteckij}
Местецкий Л. М. Непрерывная морфология бинарных изображений: фигуры,
скелеты, циркуляры. М. : Физматлит, 2009.

\bibitem{mesteckij-rejer}
Местецкий Л. М., Рейер И. Непрерывное скелетное представление 
изображения с контролируемой точностью // International Conference
Graphicon. M., 2003. С. 51–54.

\bibitem{nosov}
Носов А. В. Локализация ключевых точек кисти руки на изображении на
основе непрерывного скелета. Математические методы моделирования,
управления и анализа данных, с. 77-79, 2015.

\bibitem{nosov-2}
Носов А. В. Алгоритм распознавания жестов
рук на основе скелетной модели кисти руки // Вестник
СибГАУ. 2014. Вып. 2(54). С. 62–67.

\bibitem{ycbcr-bib}
Basilio, Jorge \& Torres, Gualberto \& Sanchez-Perez, Gabriel \& Toscano,
Karina \& Perez-Meana, Hector. Explicit image detection using
YCbCr space color model as skin detection, 2011. - p. 123-128. 

\bibitem{dix}
Dix A., Finlay J., Abowd G.D., Beale R. Human-Computer Interaction. - Third Edition, Pearson
Education Limited: 2004. - p. 857

\bibitem{graham}
Graham R. L. An efficient algorithm for determining the
convex hull of a finite planar set //Info. Pro. Lett. –
1972. – Т. 1. – С. 132-133.

\bibitem{jarvis}
Jarvis R.A. On the identification of the convex hull of a finite set of
points in the plane // Information Processing Letters, Volume 2,
Issue 1, 1973, p. 18-21.

\bibitem{jiangqin}
Jiangqin W., Wen G. A FAst Sign Word Recognition Method for Chinese Sign Language // In 
Proceedings of the Third International Conference on Advances in Multimodal Interfaces (ICMI 
'00). - Springer-Verlag, London, 2000. - p.599-606

\bibitem{MOG-1}
Kaewtrakulpong P., Bowden R.. An improved adaptive 
background mixture model for real-time tracking with
shadow detection // Video-Based Surveillance Systems, pp.
135-144. Springer, 2002.

\bibitem{kirkpatrick}
Kirkpatrick, David G.; Seidel, Raimund (1986). "The ultimate planar
convex hull algorithm?". SIAM Journal on Computing. 15 (1): 287–299.

\bibitem{otsu}
N. Otsu. A threshold selection method from gray-level histograms
(англ.) // IEEE Trans. Sys., Man., Cyber. : journal. — 1979. — Vol. 9.
— p. 62—66.

\bibitem{sanna}
Sanna A., Lamberti F., Paravati G., Henao R., Eduardo A., Manuri F. A Kinect-Based Natural
Interface for Quadrotor Control // Intelligent Technologies for Interactive Entertainment, 
Volume 78. Springer Berlin Heidelberg, 2012. - p. 48-56

\bibitem{satoshi}
Satoshi S., Keiich A. 1985. Topological Structural Analysis of
Digitized Binary Images by Border Following. Computer vision, graphics,
and image processing, 30. 

\bibitem{ten-chin}
Teh C-H and Chin Roland T. On the detection of dominant points on
digital curves. Pattern Analysis and Machine Intelligence, IEEE
Transactions on, 11(8):859–872, 1989.

\bibitem{van-vibe}
Van Droogenbroeck M., Barnich O.. Vibe: A disruptive
method for background subtraction. // In T. Bouwmans, F.
Porikli, B. Hoferlin, A. Vacavant, editors, Background
Modeling and Foreground Detection for Video Surveillance,
chapter 7. Chapman and Hall/CRC, pages 7.1-7.23, July
2014.

\bibitem{MOG-2}
Zivkovic Z., Heijden F. Efficient adaptive density
estimation per image pixel for the task of background
subtraction // Pattern recognition letters, Vol. 27(7),
pp. 773-780, 2006.

\end{thebibliography}

