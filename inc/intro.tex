\anonsection{ВВЕДЕНИЕ}
Развитие технологий, продолжающееся уже более полувека, приводит к тому, что нижняя граница
размеров процессоров уменьшается, в то время как их производительность продолжает
увеличиваться. Для рядового пользователя результатом этого является разнообразие созданных
интеллектуальных систем, используемых в повседневной жизни: от смартфонов до планшетов, от
бытовой техники до домашних роботов. Камнем преткновения становится обмен информацией между
компьютером и пользователем, растёт потребность в исследовании новых способов, более 
естественных, чем ввод данных с клавиатуры, эмулирующих бытовое общение между людьми. 
Активно развивающимся направлением решения проблемы является управление компьютером с помощью 
голосовых команд. Тем не менее, несмотря на значительные успехи в области, этот способ
применим далеко не во всех ситуациях. Другим исследуемым способом ввода данных является
использование визуальных систем: передача информации посредством мимики и жестов. Для реализации
последнего метода можно воспользоваться многими способами. 

В данной работе идёт речь о детектирование кисти руки для дальнейшей её обработки.
Детектирование позволяет находить кисть с определённой точностью, а в
дальнейшем извлекать её конфигурацию и давать её описание, что позволяет не только задавать исполнение определённых 
команд по заранее определённым жестам, но и выполнять совершенно не характерные для обычной
реализации данного метода действия.

В ходе работы исследуются различные методы и техники детектирования кисти руки человека.
Важным требованием для реализации является минимизация времени
задержки на обработку каждого кадра для комфортного использования в прикладных задачах.