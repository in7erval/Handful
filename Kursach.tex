\documentclass[a4paper,12pt]{article}         % класс документа - статья. Также report, book и 
%%%% Преамбула %%%

\usepackage[usenames,dvipsnames]{color}
\usepackage{geometry}
\geometry{left=3cm,right=1cm,top=2cm,bottom=2cm}
\usepackage[utf8]{inputenc}
\usepackage{mathtext}
\usepackage[T2A]{fontenc}
%\inputencoding{cp1251}          % тоже кодировка...
\usepackage[russian]{babel}
\usepackage[unicode]{hyperref}
\usepackage{amstext,amsmath,amssymb}
\usepackage{bm}
\usepackage[pdftex]{graphicx}
\usepackage{titlesec}
\usepackage[title,titletoc]{appendix}

\usepackage{amsfonts}           % греческие символы и, возможно, что-то ещё
\usepackage{indentfirst}        % одинаковый отступ для первого параграфа и всего остального
\usepackage{cite}               % команда /cite{1,2,7,9} даёт ссылки
\usepackage{multirow}           % пакет для объединения строк в таблице: надо указать число строк и ширину столбца
\usepackage{array}              % нужен для создания таблиц
\linespread{1.3}                % полтора интервала. Если 1.6, то два интервала
\pagestyle{plain}               % номерует страницы
\usepackage{misccorr}
\usepackage{listings} % Оформление исходного кода
\lstset{
	language=Python,
    basicstyle=\small\ttfamily, % Размер и тип шрифта
    breaklines=true, % Перенос строк
    tabsize=2, % Размер табуляции
    literate={--}{{-{}-}}2 % Корректно отображать двойной дефис
}
\usepackage{pdfpages} % Вставка PDF
\usepackage{hyphenat}

%
%% Шрифты, xelatex

%
%% Русский язык
%\usepackage{polyglossia}
%\usepackage{amssymb,amsfonts,amsmath} % Математика
%\numberwithin{equation}{section} % Формула вида секция.номер
%
%\usepackage{enumerate} % Тонкая настройка списков
%\usepackage{indentfirst} % Красная строка после заголовка
%\usepackage{float} % Расширенное управление плавающими объектами
%\usepackage{multirow} % Сложные таблицы
%
%% Формат подрисуночных записей
%\usepackage{chngcntr}
%\counterwithin{figure}{section}
%
%% Гиперссылки
\usepackage{hyperref}
\hypersetup{
    colorlinks, urlcolor={black}, % Все ссылки черного цвета, кликабельные
    linkcolor={black}, citecolor={black}, filecolor={black},
    pdfauthor={Дмитрий Юдаков},
    pdftitle={Исследование методов и разработка алгоритма детектирования, описания и извлечения конфигураций кисти человека}
}
%
%% Оформление библиографии и подрисуночных записей через точку
%\makeatletter
%\renewcommand*{\@biblabel}[1]{\hfill#1.}
%\renewcommand*\l@section{\@dottedtocline{1}{1em}{1em}}
%\renewcommand{\thefigure}{\thesection.\arabic{figure}} % Формат рисунка секция.номер
%\renewcommand{\thetable}{\thesection.\arabic{table}} % Формат таблицы секция.номер
%\def\redeflsection{\def\l@section{\@dottedtocline{1}{0em}{10em}}}
%\makeatother
%
\renewcommand{\baselinestretch}{1.4} % Полуторный межстрочный интервал
\parindent 1.25cm % Абзацный отступ
%
%\sloppy             % Избавляемся от переполнений
%\hyphenpenalty=1000 % Частота переносов
\clubpenalty=10000  % Запрещаем разрыв страницы после первой строки абзаца
\widowpenalty=10000 % Запрещаем разрыв страницы после последней строки абзаца
%

%% Списки
\usepackage{enumitem}
\setlist[enumerate,itemize]{leftmargin=12.5mm} % Отступы в списках
%
%\makeatletter
%    \AddEnumerateCounter{\asbuk}{\@asbuk}{м)}
%\makeatother
\renewcommand{\theenumi}{\arabic{enumi}}

\setlist{nolistsep} % Нет отступов между пунктами списка
%\renewcommand{\labelitemi}{--} % Маркет списка --
\renewcommand{\labelenumi}{\arabic{enumi}. } % Список второго уровня
\renewcommand{\labelenumii}{\arabic{enumi}.\arabic{enumii}.}
%% Содержание
\usepackage{tocloft}
\renewcommand{\cftsecfont}{\hspace{0pt}} % Имена секций в содержании не жирным шрифтом
\renewcommand\cftsecleader{\cftdotfill{\cftdotsep}} % Точки для секций в содержании
\renewcommand\cftsecpagefont{\mdseries} % Номера страниц не жирные
\setcounter{tocdepth}{3} % Глубина оглавления, до subsubsection
\renewcommand{\cftsecaftersnum}{.}
%
%% Формат подрисуночных надписей
\RequirePackage{caption}
\DeclareCaptionLabelSeparator{defffis}{. } % Разделитель
\captionsetup[figure]{justification=centering, labelsep=defffis, format=plain} % Подпись рисунка по центру
\captionsetup[table]{justification=centering, labelsep=defffis, format=plain, singlelinecheck=false} % Подпись таблицы слева
\addto\captionsrussian{\renewcommand{\figurename}{Рис.}} % Имя фигуры

\newcommand{\addimg}[4]{ % Добавление одного рисунка
    \begin{figure}
        \centering
        \includegraphics[width=#2\linewidth]{#1}
        \caption{#3} \label{#4}
    \end{figure}
}

\newcommand{\addimghere}[4]{ % Добавить рисунок непосредственно в это место
    \begin{figure}[H]
        \centering
        \includegraphics[width=#2\linewidth]{#1}
        \caption{#3} \label{#4}
    \end{figure}
}

\newcommand{\addtwoimghere}[4]{
	\begin{figure}[h]
		\begin{minipage}[h]{0.5\linewidth}
			\center{\includegraphics[width=0.9\linewidth]{#1} \\ а)}
		\end{minipage}
		\hfill
		\begin{minipage}[h]{0.5\linewidth}
			\center{\includegraphics[width=0.9\linewidth]{#2} \\ б)}
		\end{minipage}
		\caption{#3}
		\label{#4}
	\end{figure}
}

\newcommand{\addtwoimgherepro}[6]{
	\begin{figure}[h]
		\begin{minipage}[h]{#5\linewidth}
			\center{\includegraphics[width=1\linewidth]{#1} \\ а)}
		\end{minipage}
		\hfill
		\begin{minipage}[h]{#6\linewidth}
			\center{\includegraphics[width=1\linewidth]{#2} \\ б)}
		\end{minipage}
		\caption{#3}
		\label{#4}
	\end{figure}
}

\newcommand{\addimgsandhistshere}[6]{
	\begin{figure}[hp]
		\begin{minipage}[hp]{0.3\linewidth}
			\center{\includegraphics[width=0.9\linewidth]{#1}} \\a) 
		\end{minipage}
		\begin{minipage}[hp]{0.7\linewidth}
			\center{\includegraphics[width=0.9\linewidth]{#2}} \\б)
		\end{minipage}
		\vfill
		\begin{minipage}[hp]{0.3\linewidth}
			\center{\includegraphics[width=0.9\linewidth]{#3}} \\в)
		\end{minipage}
		\begin{minipage}[hp]{0.7\linewidth}
			\center{\includegraphics[width=0.9\linewidth]{#4}} \\г)
		\end{minipage}
		\caption{#5}
		\label{#6}
	\end{figure}
}

\newcommand{\addtwoxtwoimghere}[6]{
	\begin{figure}[hp]
		\begin{minipage}[hp]{0.5\linewidth}
			\center{\includegraphics[width=0.7\linewidth]{#1}} \\a) 
		\end{minipage}
		\begin{minipage}[hp]{0.5\linewidth}
			\center{\includegraphics[width=0.7\linewidth]{#2}} \\б)
		\end{minipage}
		\vfill
		\begin{minipage}[hp]{0.5\linewidth}
			\center{\includegraphics[width=0.7\linewidth]{#3}} \\в)
		\end{minipage}
		\begin{minipage}[hp]{0.5\linewidth}
			\center{\includegraphics[width=0.7\linewidth]{#4}} \\г)
		\end{minipage}
		\caption{#5}
		\label{#6}
	\end{figure}
}

\addto\captionsrussian{
\def\refname{СПИСОК ИСПОЛЬЗОВАННОЙ ЛИТЕРАТУРЫ}
\def\contentsname{СОДЕРЖАНИЕ}
}

%
%% Заголовки секций в оглавлении в верхнем регистре
%\usepackage{textcase}
%\makeatletter
%\let\oldcontentsline\contentsline
%\def\contentsline#1#2{
%    \expandafter\ifx\csname l@#1\endcsname\l@section
%        \expandafter\@firstoftwo
%    \else
%        \expandafter\@secondoftwo
%    \fi
%    {\oldcontentsline{#1}{\MakeTextUppercase{#2}}}
%    {\oldcontentsline{#1}{#2}}
%}
%\makeatother
%
%% Оформление заголовков
%\usepackage[compact,explicit]{titlesec}
%\titleformat{\section}{}{}{12.5mm}{\centering{\thesection\quad\MakeTextUppercase{#1}}\vspace{1.5em}}
%\titleformat{\subsection}[block]{\vspace{1em}}{}{12.5mm}{\thesubsection\quad#1\vspace{1em}}
%\titleformat{\subsubsection}[block]{\vspace{1em}\normalsize}{}{12.5mm}{\thesubsubsection\quad#1\vspace{1em}}
%\titleformat{\paragraph}[block]{\normalsize}{}{12.5mm}{\MakeTextUppercase{#1}}
%
%% Секции без номеров (введение, заключение...), вместо section*{}
\newcommand{\anonsection}[1]{
\section*{#1}
\addcontentsline{toc}{section}{#1}
}

%
%% Секции для приложений
%\newcommand{\appsection}[1]{
%    \phantomsection
%    \paragraph{\centerline{{#1}}}
%    \addcontentsline{toc}{section}{\uppercase{#1}}
%}
%
%% Библиография: отступы и межстрочный интервал
%\makeatletter
%\renewenvironment{thebibliography}[1]
%    {\section*{\refname}
%        \list{\@biblabel{\@arabic\c@enumiv}}
%           {\settowidth\labelwidth{\@biblabel{#1}}
%            \leftmargin\labelsep
%            \itemindent 16.7mm
%            \@openbib@code
%            \usecounter{enumiv}
%            \let\p@enumiv\@empty
%            \renewcommand\theenumiv{\@arabic\c@enumiv}
%        }
%        \setlength{\itemsep}{0pt}
%    }
%\makeatother

\usepackage{listings}
\usepackage{color}
\usepackage{minted}
\usepackage{python}
%\definecolor{dkgreen}{rgb}{0,0.6,0}
%\definecolor{gray}{rgb}{0.5,0.5,0.5}
%\definecolor{mauve}{rgb}{0.58,0,0.82}
%
\lstset{ %
language=Python,                 % выбор языка для подсветки (здесь это С)
basicstyle=\small\sffamily, % размер и начертание шрифта для подсветки кода
numbers=left,               % где поставить нумерацию строк (слева\справа)
numberstyle=\tiny,           % размер шрифта для номеров строк
stepnumber=1,                   % размер шага между двумя номерами строк
numbersep=5pt,                % как далеко отстоят номера строк от подсвечиваемого кода
backgroundcolor=\color{white}, % цвет фона подсветки - используем \usepackage{color}
showspaces=false,            % показывать или нет пробелы специальными отступами
showstringspaces=false,      % показывать или нет пробелы в строках
showtabs=false,             % показывать или нет табуляцию в строках
frame=single,              % рисовать рамку вокруг кода
tabsize=2,                 % размер табуляции по умолчанию равен 2 пробелам
captionpos=t,              % позиция заголовка вверху [t] или внизу [b] 
breaklines=true,           % автоматически переносить строки (да\нет)
breakatwhitespace=false, % переносить строки только если есть пробел
escapeinside={\%*}{*)}   % если нужно добавить комментарии в коде
}


  \makeatletter
  \renewcommand*{\@biblabel}[1]{\hfill#1.}
  \makeatother


    
\sloppy

%\setcounter{page}{4} % Начало нумерации страниц
\begin{document}

\setcounter{page}{2}            % Нумерация страниц начинается с "2"

\newpage                        % Начинает текст с новой страницы
\tableofcontents                % Автоматическое создание оглавления по названиям разделов, подразделов и т.п.

\newpage

\anonsection{Введение}
Развитие технологий, продолжающееся уже более полувека, приводит к тому, что нижняя граница
размеров процессоров уменьшается, в то время как их производительность продолжает
увеличиваться. Для рядового пользователя результатом этого является разнообразие созданных
интеллектуальных систем, используемых в повседневной жизни: от смартфонов до планшетов, от
бытовой техники до домашних роботов. Камнем преткновения становится обмен информацией между
компьютером и пользователем, растёт потребность в исследовании новых способов, более 
естественных, чем ввод данных с клавиатуры, эмулирующих бытовое общение между людьми. 
Активно развивающимся направлением решения проблемы является управление компьютером с помощью 
голосовых команд. Тем не менее, несмотря на значительные успехи в области, этот способ
применим далеко не во всех ситуациях. Другим исследуемым способом ввода данных является
использование визуальных систем: передача информации посредством мимики и жестов. Для реализации
последнего метода можно воспользоваться многими способами. 

В данной работе идёт речь о поиске так называемых ""особых""\ точек на кисти руки 
для дальнейшей их обработки. Поиск таких точек позволяет извлекать конфигурацию кисти
и давать её описание, что позволяет не только задавать исполнение определённых 
команд по заранее определённым жестам, но и выполнять совершенно не характерные для обычной
реализации данного метода действия.

В ходе исследования разрабатывается алгоритм детектирования кисти руки человека и её описания
с помощью "особых"\ точек. Важным требованием для реализации является минимизация времени
задержки на обработку каждого кадра для комфортного использования в прикладных задачах.

Поставленной {\bf целью} является создание программы для детектирования и извлечения 
конфигурации кисти руки человека с помощью языка высокого уровня Python.

Для достижения цели необходимо решить ряд {\bf задач}:
\begin{enumerate}
	\item Сделать обзор теоретического материала по детектированию кисти руки человека и 
её последующего описания с помощью "особых"\ точек.
	\item Найти и описать наиболее популярные методы обнаружения кисти человека и выделить из них самые эффективные.
	\item Ознакомиться с методами извлечения конфигураций кисти человека.
	\item Реализовать несколько алгоритмов данной задачи и провести их сравнение.
\end{enumerate}
\section{\nohyphens{МЕТОДЫ ДЕТЕКТИРОВАНИЯ КИСТИ ЧЕЛОВЕКА В СИСТЕМАХ ЧЕЛОВЕКО-МАШИННОГО ВЗАИМОДЕЙСТВИЯ}}

Человеко-машинное взаимодействие (Human-computer interaction - HCI) - это междисциплинарное
научное направление, изучающее взаимодействие между людьми и машинами. Предметом HCI является
изучения, планирование и разработка методов взаимодействия человека с машиной, где в роли машины
может выступать персональный компьютер, компьютерная система больших масштабов, система 
управления процессами и т.д. \cite{dix}. Под взаимодействием понимается любая коммуникация между
человеком и машиной. Одним из методов HCI, получившим широкое распространение в последние годы,
является взаимодействие, основанное на жестах человека \cite{jiangqin, sanna}. 

Задачу распознавания жестов руки можно разделить на подзадачи:
\begin{enumerate}
	\item Отделение кисти руки от остальной части изображения.
	\item Построение контура кисти.
	\item Нахождение ключевых точек на кисти.
	\item Классификация жеста исходя из статического или динамического расположения точек.
\end{enumerate}

Первая подзадача, а именно детектирование кисти человека в кадре является ключевой, поскольку
от качества её решения зависит качество выполнения остальных подзадач.
Рассмотрим первую подзадачу, а именно детектирование кисти человека в кадре.

Существует множество решений этой подзадачи. Наиболее популярными из них являются 
отделение фона изображения, распознавание цвета 
кожи в кадре и метод Оцу. 
Подробно рассмотрим каждое из них.

\subsection{Отделение фона изображения}

В данном решении принимается, что в кадре движется только рука, а остальные части
тела, включая фон, остаются неподвижными. Таким образом, если вначале инициализировать фон как
$bgr(x, y)$, а новое изображение с жестом рассматривать как $fgr(x, y)$, то изолированный жест
можно принять как разность между этими изображениями: 
\begin{equation} gst_i(x,y)=fgr_i(x,y)-bgr(x,y). \label{first}\end{equation}

Полученный разностный жест переднего плана преобразуется в бинарное изображение, устанавливая
соответствующий порог. Поскольку фон не является полностью статичным, например, если камера
удерживается в руках оператора, то добавляется шумовая часть. Чтобы получить изображение руки
без шумов, этот метод сочетается с распознаванием кожи человека. Чтобы удалить этот шум, 
применяется анализ связанных компонентов, чтобы заполнить пустоты, если применяется заливка
какой-либо области, а также для получения чётких краёв применяется морфологическая обработка.

Недостаток данного решения состоит в том, что довольно сложно отделить фон, даже если он задан,
поскольку не ясно какие из пикселей изменились, а какие остались прежними из-за тени, смещения
фокуса, изменения экспозиции и т.д. Даже если зафиксировать все параметры камеры, то тень так
или иначе испортит качество решения.

Приведём два примера отделения кисти от фона изображения. На рис. \ref{pix1} изображены два
изображения, одно из которых является инициализированным фоном, а второе -- кисть на этом фоне.

\addtwoimghere{pix/BG_result}{pix/FG_result}{Инициализированный фон (а) и изображение
кисти на нём (б).}{pix1}
Для отделения кисти первым способом воспользуемся встроенной в библиотеку OpenCV функцией
{\tt absdiff()}.

Второй способ будет заключаться в следующем. Пусть даны два пикселя 
\begin{equation}
	p_1 = (r_1, g_1, b_1)~~и~~p_2 = (r_2, g_2, b_2),
	\label{pixels_bg}
\end{equation}
тогда различие между ними будем определять как
\begin{equation}
	D = \sqrt{(r_2 - r_1)^2 + (g_2 - g_1)^2 + (b_2 - b_1)^2}.
	\label{distance}
\end{equation}
Затем создаём битовую маску $M$, в которой значения определяются как
\begin{equation}
	M_{(i,j)}=\left\{\begin{aligned}
	1,~& D > Threshold\\
	0,~& D \leq Threshold
\end{aligned}\right. 
\label{bite-mask}
\end{equation}

Результаты работы первого и второго метода представлены на рис. \ref{pix2}.
\addtwoimghere{pix/result}{pix/result1}{Результат первого метода (а) и второго (б).}{pix2}

Как можно видеть, оба метода не совсем точно отделяют нужный объект от фона, поэтому 
рассмотрим метод распознавания кожи в HSV и YCbCr цветовых моделях.

\newpage

\subsection{Метод распознавания кожи в HSV и YCbCr цветовых моделях}

Для того, чтобы детектировать цвет кожи на изображениях очень часто применяются 
различные цветовые модели, а именно HSV и YCbCr.

{\bf HSV} ({\it Hue, Saturation, Value}) или {\bf HSB} ({\it Hue, Saturation, Brightness}) --
цветовая модель, в которой координатами цвета являются:
\begin{enumerate}
	\item {\bf H}ue -- цветовой тон, (например, красный, зелёный или сине-голубой). Варьируется 
в пределах 0-360\textdegree, однако иногда приводится к диапазону 0-100 или 0-1.
	\item {\bf S}aturation -- насыщенность. Варьируется в пределах 0-100 или 0-1. Чем больше
этот параметр, тем "чище"\ цвет, поэтому иногда называют {\it чистотой цвета}. А чем ближе этот
параметр к нулю, тем ближе цвет к нейтральному серому.
	\item {\bf V}alue или {\bf B}rightness -- яркость. Также задаётся в пределах 0-100 или 0-1.
\end{enumerate}

{\bf YCbCr} -- семейство цветовых пространств, которые используются для передачи цветных
изображений в компонентном видео и цифровой фотографии. Является частью рекомендации 
МСЭ-R ВT.601 при разработке стандарта видео Всемирной цифровой организации и фактически
является масштабированной и смещённой копией YUV. 

{\bf Y} -- компонента яркости, {\bf Cb} -- синий компонент цветности, 
{\bf Cr} -- красный компонент цветности.

Пример изображения в HSV изображён на рисунке \ref{rgb-hsv-pic}.
\addtwoimgherepro{pix/FG_result}{pix/hsv_example}{Изображение в RGB (а) и HSV (б).}
{rgb-hsv-pic}{0.39}{0.39}

Пример изображения в $YC_BC_R$ изображен на рисунке \ref{rgb-ycbcr-pic}.
\addtwoimgherepro{pix/FG_result}{pix/ycbcr_example}{Изображение в RGB (а) и $YC_BC_R$ (б).}
{rgb-ycbcr-pic}{0.39}{0.39}

Для того чтобы отделить кожу от остальной части изображения, 
используют определённые диапазоны составляющих цветовых моделей, 
которые находятся эмпирически или итеративно. В первом случае путём 
проб и ошибок подбираются значения составляющих. Можно заметить,
что при данном подходе кожа будет иметь разные цветовые диапазоны
при разном освещении. Покажем это. Зададим диапазон для 
изображения в HSV цветовой модели как

$$
\begin{aligned}
	&0 \leq H \leq 200, \\
	&15 \leq S \leq 255, \\
	&80 \leq V \leq 255,
\end{aligned}
$$
и получим результат, представленный на рис. \ref{hsv-del-pic}.

Можно видеть, что на первом изображении фон отделился практически 
идеально, но на втором видны фрагменты фона.

\newpage

\addtwoxtwoimghere{pix/image3}{pix/hsv_del1}{pix/image7}{pix/hsv_del2}{Изображения с удалённым
фоном в HSV.}{hsv-del-pic}

Для цветовой модели YCbCr в статье \cite{ycbcr-bib} предложили два 
варианта диапазона компонент (\ref{ycbcr-diap1}) и (\ref{ycbcr-diap2}):

\begin{equation}
	\begin{aligned}
		&80 < Y \leq 255, \\
		&85 < C_b < 135, \\
		&135 < C_r < 180 
	\end{aligned}
	\label{ycbcr-diap1}
\end{equation}

\begin{equation}
	\begin{aligned}
		&Y \in \forall,\\
		&77 \leq C_b \leq 127, \\
		&133 \leq C_r \leq 173 
	\end{aligned}
	\label{ycbcr-diap2}
\end{equation}

Сравнение двух вариантов представлено на рисунке \ref{ycbcr-del-pic}.

\addtwoxtwoimghere{pix/ycbcr_del1_1}{pix/ycbcr_del1_2}
{pix/ycbcr_del2_1}{pix/ycbcr_del2_2}{Изображения с удалённым
фоном в YCbCr.}{ycbcr-del-pic}

Легко видеть, что оба диапазона работают не идеально, но первый 
показал себя намного лучше. 

Таким образом, можно сделать вывод, что для детектирования цвета кожи 
лучше брать изображение в цветовой модели HSV. 

Рассмотрим следующий метод отделения фона изображения, а именно метод
Оцу.

\subsection{Метод Оцу}

В 1979 году Нобуюки Оцу опубликовал статью \cite{otsu} метода порогового 
разделения, основываясь на гистограмме серых цветов изображения. 

Метод Оцу -- это алгоритм вычисления порога бинаризации для полутонового
изображения, используемый в области компьютерного распознавания образов
и обработки изображений для получения чёрно-белых изображений. Алгоритм
позволяет разделить пиксели двух классов ("полезные"\ и "фоновые"), 
рассчитывая такой порог, чтобы внутриклассовая дисперсия была
минимальной. Для того чтобы упростить алгоритм, используется
гистограмма монохромного изображения.
 
Диаграммы для изображений изображены на рис. \ref{histograms}. 
Как видно из рисунков, на данных изображениях довольно
проблематично отделить фон от изображения какой-то единственной 
пороговой величиной, поэтому и алгоритм Оцу на таких изображениях 
сработает не очень хорошо, как это показано на рис. \ref{otsu-ex}. 
Реализация данного алгоритма и построение гистограмм изображений
находятся в приложении А.

\addimgsandhistshere{pix/image3}{pix/histogram_for_3}
{pix/image7}{pix/histogram_for_7}{Изображение 1 и 2
и их гистограммы.}{histograms}

\newpage

\addtwoimghere{pix/otsu_exmpl_for_3}{pix/otsu_exmpl_for_7}{Результат работы программы с алгоритмом Оцу}{otsu-ex}

\subsection{Mixture of Gaussians}

Стандартным подходом к построению модели фона, использующимся для многих
прикладных задач, является смесь гауссовых распределений (Mixture of Gaussians,
MOG) \cite{MOG-1, MOG-2}. Чаще всего для каждого пикселя
текущего кадра с номером $n$ строится функция плотности вероятности 
$P_n=P(\nu_n(p))$, и MOG используется именно этот подход. Предполагается, что 
для каждого пикселя текущего изображения она может быть представлена смесью
нормальных распределений, где $G$ -- их число в смеси.

Для инициализации гауссиан для каждого пикселя чаще всего применяют либо
EM-алгоритм (Expectation-maximization algorithm), либо k-means, что
достаточно затратно в вычислительном плане. Число входящих в смесь распределений
$G$ обычно принимают равным от 3 до 5. Также существует подход, позволяющий
автоматически подбирать необходимое количество гауссиан \cite{MOG-2}. 

В библиотеке OpenCV данный метод реализуется с помощью встроенной функции
{\tt createBackgroundSubtractorMOG2()}. Данная функция обучается на изображениях
каждого кадра и лучше всего работает в режиме последовательности, то есть на
видео. Результат работы метода MoG представлен на рисунке
\ref{mog-example-img}.

\newpage

\addimghere{pix/mog_result}{1}
{Результат отделения фона от изображения с помощью метода MoG.}{mog-example-img}

В коде видно, что сначала происходит обучение на первых изображениях фона, 
а затем добавляется новый кадр, на котором стирается фон.

\bigskip

Рассмотрев наиболее популярные методы отделения кисти человека от фона изображения, можно сделать
вывод, что некоторые из методов работают наилучшим образом при специфических условиях, поэтому
необходимо их дополнительное исследование. 










\section{Методы построения контура объекта на изображении}

Отслеживание границ -- один из основных методов обработки оцифрованных
двоичных изображений. Он производит последовательность координат или 
цепных кодов от границ между связным компонентом

\subsection{Выделение границ с помощью оператора Кэнни}

Оператор Кэнни (детектор границ Кэнни или алгоритм Кэнни) -- оператор
обнаружения границ изображения. Кэнни изучил математическую проблему
получения фильтра, оптимального по критериям выделения, локализации и
минимизации нескольких откликов одного края. Он показал, что искомый
фильтр является суммой четырёх экспонент. Он также показал, что этот
фильтр может быть хорошо приближен первой производной Гауссианы. Кэнни
ввёл понятие {\it подавление немаксимумов} (Non-Maximum Suppression), 
которое означает, что пикселями границ объявляются пиксели, в которых
достигается локальный максимум градиента в направлении вектора
градиента.

Целью Кэнни было разработать оптимальный алгоритм обнаружения границ, 
удовлетворяющий трём критериям:
\begin{itemize}
	\item хорошее обнаружение (повышение отношения сигнал/шум);
	\item хорошая локализация (правильное определения положения
границы);
	\item единственный отклик на одну границу.
\end{itemize}

Из этих критериев затем строится целевая функция стоимости ошибок, 
минимизацией которой находится "оптимальный"\ линейный оператор для
свёртки с изображением.

Первым этапом алгоритма является {\it сглаживание}, то есть размытие
изображения для удаления шума. Оператор Кэнни использует фильтр, 
который может быть хорошо приближен к первой производной гауссианы.
Для $\sigma=1.4$:

\begin{equation}
	B = \frac{1}{159}
\begin{bmatrix}
	2 & 4 & 5 & 4 & 2\\
	4 & 9 & 12 & 9 & 4\\
	5 & 12 & 15 & 12 & 5\\
	4 & 9 & 12 & 9 & 4\\
	2 & 4 & 5 & 4 & 2
\end{bmatrix}
\cdot A.
\label{canny-1}
\end{equation}

Следующим этапом является {\it поиск градиентов}. Границы отмечаются 
там, где градиент изображения приобретает максимальное значение. Они 
могут иметь различное направление, поэтому алгоритм Кэнни использует
четыре фильтра для обнаружения горизонтальных, вертикальных и 
диагональных ребер в размытом изображении.
\begin{equation}
    G = \sqrt{G_x^2 + G_y^2},
    \Theta = \arctan (\frac{G_y}{G_x}).
	\label{canny-2}
\end{equation}

Угол направления вектора градиента при этом округляется и может 
принимать такие значения: 0\textdegree, 45\textdegree, 90\textdegree,
135\textdegree.

Затем происходит {\it подавление немаксимумов}, когда только локальные
максимумы отмечаются как границы, {\it двойная пороговая филтьтрация}
-- потенциальные границы определяются порогами и {\it трассировка
области неоднозначности}, когда итоговые границы определяются путём 
подавления всех краёв, не связанных с определёнными (сильными) 
границами.

Перед применением детектора обычно преобразуют изображение в оттенки
серого, чтобы уменьшить вычислительные затраты. 

С помощью следующего кода построим примеры работы алгоритма Кэнни:

\begin{minted}[mathescape, 
				linenos, 
				gobble=0, 
				frame=lines,
				framesep=2mm]{python}
import cv2
from matplotlib import pyplot as plt

CV2_CVT = [cv2.COLOR_BGR2RGB, cv2.COLOR_BGR2GRAY,
           cv2.COLOR_BGR2HSV, cv2.COLOR_BGR2YCR_CB]
TITLES = ['RGB', 'GRAY', 'HSV', 'YCbCr']
def plot_images(img_name):
    fig, axes = plt.subplots(2, 2)
    fig.set_figwidth(10)
    fig.set_figheight(10)
    axes = axes.flatten()
    img = cv2.imread(img_name)
    img_blur = cv2.GaussianBlur(img, (5, 5), 2)
    for i, ax in enumerate(axes):
        ax.imshow(canny_wrapper(img_blur, CV2_CVT[i]),
                  cmap='gray')
        ax.set_title(TITLES[i])
    for ax in axes:
        ax.set_xticks([])
        ax.set_yticks([])
    plt.show()
    fig.savefig(f'canny_image1.png')
plot_image('image1.jpg')
\end{minted}

\newpage

Пример работы оператора в цветовых моделях RBG, GRAY, HSV и YCbCr 
показан на рис. \ref{canny-img1}.

\addtwoimgherepro{pix/image1}{pix/canny_image1}{Пример 1 работы
оператора Кэнни.}{canny-img1}{0.3}{0.8}

Можно видеть, что алгоритм Кэнни довольно неплохо позволяет определить
границы объекта. Но для задачи извлечения конфигурации кисти 
необходимо извлекать особые точки изображения, а этот метод лишь 
выделяет границы на изображении, не определяя сам контур. В связи с
этим необходимо рассмотреть топологический структурный анализ цифрового
бинарного изображения с помощью отслеживания границ.

\subsection{Топологический структурный анализ цифрового бинарного
изображения с помощью отслеживания границ.}

Этот метод был разработан Сатоши Сузуки и Кейчи Эйбом в 1985 году
\cite{satoshi}. Алгоритм предполагает нахождение контуров с учетом
вложенности, то есть способен определить, когда в контур одного объекта
вложен другой. Реализация данного алгоритма лежит в основе функции
{\tt findContours()} в библиотеке OpenCV, предназначенной для 
исследования и решения задач, связанных с компьютерным зрением. 

\subsubsection{Обзор функции {\tt findContours()}.}

Сигнатура функции {\tt findContours()}, реализованной в библиотеке
OpenCV для языка Python выглядит следующим образом:

\begin{minted}[mathescape, 
				linenos, 
				gobble=0, 
				frame=lines,
				framesep=2mm]{python}
def findContours(image, mode, method, contours=None, 
			hierarchy=None, offset=None):
	pass
\end{minted}

Опишем каждый из параметров:
\begin{itemize}
	\item {\tt image} -- Исходное 8-битное бинарное изображение.
Для преобразование можно использовать функции {\tt inRange()}, 
{\tt threshold()}, {\tt adaptiveThreshold()} и ранее использованная 
функция, реализующая алгоритм Кэнни -- {\tt Canny()}. Если {\tt mode}
является {\tt RETR\_CCOMP} или {\tt RETR\_FLOODFILL}, то изображение 
может быть 32-битным.
	\item {\tt mode} -- Режим поиска контуров. Может принимать 
следующие значения:
		\begin{itemize}
			\item {\tt RETR\_EXTERNAL} -- поиск только внешних
контуров.
			\item {\tt RETR\_LIST} -- поиск всех контуров без установки
их отношения.
			\item {\tt RETR\_CCOMP} -- поиск всех контуров и
организация их в иерархию, состоящую из двух уровней: на верхнем уровне
находится внешние границы компонент, а на втором -- границ "дыр". Если
внутри "дыры"\ есть другой контур, то он устанавливается на верхнем
уровне.
			\item {\tt RETR\_TREE} -- поиск всех контуров и установка
их иерархии вложенных контуров.
			\item {\tt RETR\_FLOODFILL}
		\end{itemize}
	\item {\tt method} -- Метод аппроксимации контуров. Может принимать
следующие значения:
		\begin{itemize}
			\item {\tt CHAIN\_APPROX\_NONE} -- без аппроксимации, 
хранятся абсолютно все точки контура. Это означает, что две 
последовательные точки $(x_1,y_1)$ и $(x_2,y_2)$ контура будут
либо горизонтальными, либо вертикальными, либо диагональными соседями, 
то есть $max(|x_1-x_2|, |y_1-y_2|) = 1$.
			\item {\tt CHAIN\_APPROX\_SIMPLE} -- сжатие горизонтальных, 
вертикальных и диагональных сегментов, и хранение только их точек
концов. Например, контур в виде прямоугольника будет описан только его
четырьмя точками (вершинами).
			\item {\tt CHAIN\_APPROX\_TC89\_L1} / 
{\tt CHAIN\_APPROX\_TC89\_KCOS} -- применение двух подходов алгоритма
Тена и Чина аппроксимации цепи \cite{ten-chin}.
		\end{itemize}
	\item {\tt contours} -- Найденные контуры.
	\item {\tt hierarchy} -- ({\it опционально})  Вектор, хранящий
информацию о топологии изображения. Количество элементов вектора равно
количеству найденных контуров. Для каждого $i-$того контура {\tt
contours[i]} элемент:
		\begin{itemize}
			\item {\tt hierarchy[i][0]} = индексу следующего контура 
на текущем уровне, 
			\item {\tt hierarchy[i][1]} = индексу предыдущего контура
на текущем уровне,
			\item {\tt hierarchy[i][2]} = индексу первого контура на
на вложенном уровне,
			\item {\tt hierarchy[i][3]} = индексу родительского
контура.
		\end{itemize}
Если для $i-$того контура нет следующего, предыдущего, родительского 
и вложенного, то соответствующие элементы являются
отрицательными.
	\item {\tt offset} -- ({\it опционально}) Величина сдвига каждой
точки контура. Полезно, если контуры вычисляются из ROI (англ. Region
Of Interest, область интереса) изображения и точки контура должны 
анализироваться в отношении к всему изображению.

\end{itemize}

Для отрисовки контуров, полученных с помощью данной функции, удобно
пользоваться функцией {\tt drawContours()}. Рассмотрим её поподробнее.

\subsubsection{Обзор функции {\tt drawContours()}.}

Сигнатура функции {\tt findContours()}, реализованной в библиотеке
OpenCV для языка Python выглядит следующим образом:

\begin{minted}[mathescape, 
				linenos, 
				gobble=0, 
				frame=lines,
				framesep=2mm]{python}
def drawContours(image, contours, contourIdx, color, thickness=None,
		lineType=None, hierarchy=None, maxLevel=None, offset=None):
	pass
\end{minted}

Параметры:
\begin{itemize}
	\item {\tt image} -- Изображение, на котором будут отрисовываться
контуры.
	\item {\tt contours} -- Все контуры.
	\item {\tt contourIdx} -- Параметр, обозначающий индекс контура
для отрисовки. Если отрицателен, то будут отрисованы все контуры.
	\item {\tt color} -- Цвет контуров.
	\item {\tt thickness} -- Толщина контуров. Если отрицателен, то
контуры будут закрашены полностью вместе с внутренним пространством.
	\item {\tt lineType} -- Тип соединения линий ({\tt FILLED}, 
{\tt LINE\_4}, {\tt LINE\_8}, {\tt LINE\_AA}).
	\item {\tt hierarchy} -- ({\it опционально}) Информация о иерархии 
контуров. Необходим, если нужно отрисовать только несколько контуров 
(см. {\tt maxLevel}).
	\item {\tt maxLevel} -- Максимальный уровень контуров для
отрисовки. Если {\tt maxLevel} = 
	\begin{itemize} 
		\item 0, то будут отрисованы только специфичные 
контуры,
		\item 1, то будут отрисованы все контур(ы) и все в них
вложенные,
		\item 2, то функция отрисовывает все контуры, все в них 
вложенные, все вложенные во вложенные контуры, и так далее. 
	\end{itemize}
Этот параметр игнорируется, если не задан параметр {\tt hierarchy}, то
есть \linebreak{\tt hierarchy = None}.
	\item {\tt offset} -- ({\it опционально}) Сдвиг контуров для отрисовки:
$\texttt{offset} = (dx, dy)$
\end{itemize}

\bigskip

Поскольку функция {\tt findContours()} находит все контуры на 
изображении, то нужно среди них выбрать один единственный. 
Пусть кисть занимает наибольшее пространство на изображении и, 
соответственно, имеет наибольший контур. Среди всех контуров будем 
отбирать тот, {\it площадь} которого имеет наибольшее значение. 
Площадь контура будем находить с помощью {\it формулы площади Гаусса}.

\subsubsection{Формула площади Гаусса.}

Формула площади Гаусса (формула землемера или формула шнурования или
алгоритм шнурования) — формула определения площади простого
многоугольника, вершины которого заданы декартовыми координатами на
плоскости. В формуле векторным произведением координат и сложением
определяется площадь области, охватывающей многоугольник, а затем из
нее вычитается площадь окружающего многоугольника, что дает площадь
многоугольника внутри. 

Формула может быть представлена следующим выражением:
\begin{equation}
\begin{aligned}
	&S = \frac{1}{2} \left| 
	\sum_{i=1}^{n-1}{x_i y_{i+1}}+x_n y_1 -
	\sum_{i=1}^{n-1}{x_{i+1} y_i} - x_1 y_n
	\right|=\\
	&=\frac{1}{2} \left| x_1 y_2 + x_2 y_3 + 
	\dots + x_{n-1} y_n + x_n y_1 - x_2 y_1
	-x_3 y_2 - \dots - x_n y_{n-1} - x_1 y_n\right|,
\end{aligned}
\label{gauss-square-equation}
\end{equation}
где

$S$ -- площадь многоугольника,

$n$ -- количество сторон многоугольника,

$(x_i, y_i), i=\overline{1,n}$ -- координаты вершин многоугольника.

Другие представления этой же формулы:
\begin{equation}
\begin{aligned}
	&S = \frac{1}{2} \left| 
	\sum_{i=1}^n{x_i (y_{i+1}-y_{i-1})} \right| =
	\frac{1}{2} \left| 
	\sum_{i=1}^n{y_i (x_{i+1}-x_{i-1})} \right| =\\
	&= \frac{1}{2} \left| 
	\sum_{i=1}^n{x_i y_{i+1} - x_{i+1} y_i} \right| =
	\frac{1}{2} \left| 
	\sum_{i=1}^n{det
	\begin{pmatrix}
		x_i & y_i \\
		x_{i+1} & y_{i+1}
	\end{pmatrix}
	} \right|,
\end{aligned}
\label{gauss-square-equation-1}
\end{equation}
где

$x_{n+1}=x_1,~x_0=x_n$,

$y_{n+1}=y_1,~y_0=y_n$.

\bigskip

Рассмотрим пример использования данных функций. Сравним результат
поиска контуров с предварительной обработкой изображения, 
заключающейся в отделении объекта от фона, и без обработки (только с 
переводом изображения в черно-белый формат). С помощью следующего
кода получим результаты, представленные на рис. \ref{contours-img-ex}.

\begin{minted}[mathescape, 
				linenos, 
				gobble=0, 
				frame=lines]{python}
import cv2

def process_image(image):
    blurred = cv2.GaussianBlur(image, (3, 3), 3)
    _, thresh1 = cv2.threshold(blurred, 0, 255,
        cv2.THRESH_BINARY_INV + cv2.THRESH_OTSU)
    return thresh1
def find_contours_otsu(filename: str, with_processing: bool = False):
    img = cv2.imread(filename)
    img_gray = cv2.cvtColor(img, cv2.COLOR_BGR2GRAY)
    image_to_contours = process_image(img_gray) if with_processing else img_gray
    contours, hierarchy = cv2.findContours(image_to_contours.copy(),
                    cv2.RETR_TREE, cv2.CHAIN_APPROX_NONE)
    print(f'Contours count = {len(contours)}')
    # находим контур с максимальной площадью
    cnt = max(contours, key=lambda x: cv2.contourArea(x))
    print(f'Contour area = {cv2.contourArea(cnt)}')
    print(f'Points of contour count = {len(cnt)}')
    cv2.drawContours(img, [cnt], -1, (0, 255, 0), 3)
    cv2.imwrite('contours_' + 
      ('processing_' if with_processing else '') 
      + 'example.png', img)
find_contours_otsu('image1.jpg')
find_contours_otsu('image1.jpg', with_processing=True)
\end{minted}

\newpage

\addtwoimghere{pix/contours_otsu_example}
{pix/contours_otsu_processing_example}
{Пример поиска контура на изображении без предварительной обработки
(а) и с предварительной обработкой с помощью метода Оцу (б).}
{contours-img-ex}

Можно видеть, что пороговая бинаризация изображения методом Оцу 
позволила с высокой точностью определить истинное расположение контура
ладони. Это означает, что для дальнейшего исследования контуров
необходима предобработка изображения.



\section{\centering МЕТОДЫ НАХОЖДЕНИЯ КЛЮЧЕВЫХ ТОЧЕК НА КИСТИ}

Особую сложность представляет задача 
определения положения ключевых точек на кисти человека.
Для этой нетривиальной задачи было предложено
несколько методов её решения. Рассмотрим каждый из них.

\subsection{Нахождение точек путём определения дефектов 
выпуклости}

В предыдущем разделе был рассмотрен метод построения контура
кисти с помощью топологического структурного анализа
цифрового бинарного изображения с помощью отслеживания
границ. Этот метод хорош тем, что позволяет позволяет
определить точные координаты точек контура. С помощью этих
точек существует возможность построить {\it выпуклую
оболочку} или их {\it наименьшее множество}.

Выпуклой оболочкой множества $X$ называется наименьшее
выпуклое множество, содержащее $X$. 
«Наименьшее множество» здесь означает наименьший элемент по
отношению к вложению множеств, то есть такое выпуклое
множество, содержащее данную фигуру, что оно содержится в
любом другом выпуклом множестве, содержащем данную фигуру.

Рассмотрим основные алгоритмы построения выпуклой оболочки.

\subsubsection{Алгоритм Грэхема}

Алгоритм Грэхема\cite{graham} — алгоритм построения выпуклой
оболочки в двумерном пространстве. В этом алгоритме задача о
выпуклой оболочке решается с помощью стека, сформированного
из точек-кандидатов. Все точки входного множества заносятся в
стек, а потом точки, не являющиеся вершинами выпуклой
оболочки, со временем удаляются из него. По завершении работы
алгоритма в стеке остаются только вершины оболочки в порядке
их обхода против часовой стрелки.

Временная сложность алгоритма = O($n\log{n}$).

{\bf Описание алгоритма:}

Пусть точки $p=[p_0,...p_{n-1}]$ -- входной массив точек.
\begin{enumerate}
	\item Найти самую "нижнюю"\ точку в массиве (ту, в которой
наименьшая среди всех координата $y$). Если таких точек несколько, то
среди них выбрать точку с наименьшей координатой $x$. Найденная точка
$P_0$ является первой точкой выпуклой оболочки;
	\item Перебрать остальные $n-1$ точек и отсортировать их по 
полярному углу относительно $P_0$ в направлении против часовой стрелки.
Если полярный угол нескольких точек одинаков, то выбрать ближайшую к 
$P_0$;
	\item После сортировки, проверить, есть ли точки с одинаковым 
полярным углом. Если да, то удалить все эти точки, кроме ближайшей к
$P_0$. Положить размер нового массива равным $m$;
	\item Если $m < 3$, то алгоритм прерывается. Выпуклую оболочку 
определить невозможно;
	\item Создать пустой стек $S$ и положить в него первые три точки 
нового массива $p_0, p_1, p_2$;
	\item Для каждой из оставшихся $m-3$ точек:
	\begin{enumerate}
		\item Удаляем точки из стека пока ориентация трёх следующих
точек не против часовой стрелки (они не совершают левый поворот):
точка наверху стека, точка следующая после неё и текущая точка $p_i$;
		\item Добавляем точку $p_i$ в $S$;
	\end{enumerate}
	\item Стек S содержит точки выпуклой оболочки.
\end{enumerate}

Реализация алгоритма Грэхема на языке Python находится в приложении Б.
Протестировав алгоритм на случайном наборе
точек, получаем результат, представленный на рисунке \ref{graham-ex}.

\addimghere{pix/graham_test}{0.9}{Сгенерированные точки (слева) и их 
выпуклая оболочка, найденная с помощью алгоритма Грэхема (справа).}
{graham-ex}

\subsubsection{Алгоритм Джарвиса}

Алгоритм Джарвиса \cite{jarvis} (или алгоритм обхода Джарвиса, или алгоритм
заворачивания подарка) определяет последовательность элементов
множества, образующих выпуклую оболочку для этого множества. Метод
можно представить как обтягивание верёвкой множества вбитых в доску
гвоздей. Алгоритм работает за время  O($nh$), где $n$ -- общее число 
точек на плоскости, а $h$ -- число точек в выпуклой оболочке. В 
худшем случае -- O($n^2$), когда все точки попадают в выпуклую
оболочку.

{\bf Описание алгоритма:}

Пусть дано множество точек $P = \{p_1, p_2, \ldots, p_n\}$. В качестве
начальной берётся самая левая нижняя точка $p_1$ (её можно найти за
$O(n)$ обычным проходом по всем точкам), она точно является вершиной
выпуклой оболочки. Следующей точкой ($p_2$) берём такую точку, 
которая имеет наименьший положительный полярный угол относительно точки
$p_1$ как начала координат. После этого для каждой точки $p_i$ 
($2 < i \leq |P|$) против часовой стрелки ищется такая точка $p_{i+1}$,
путём нахождения за $O(n)$ среди оставшихся точек 
(+ самая левая нижняя),
в которой будет образовываться наибольший угол между прямыми 
$p_{i-1}p_i$ и $p_ip_{i+1}$. Она и будет следующей вершиной выпуклой
оболочки. Сам угол при этом не ищется, а ищется только его косинус 
через скалярное произведение между лучами $p_{i-1}p_i$ и 
$p_ip'_{i+1}$, где $p_i$ -- последний найденный минимум, $p_{i-1}$ -- 
предыдущий минимум, а $p'_{i+1}$ -- претендент на следующий минимум.
Новым минимумом будет та точка, в которой косинус будет принимать
наименьшее значение (чем меньше косинус, тем больше его угол).
Нахождение вершин выпуклой оболочки продолжается до тех пор, пока 
$p_{i+1} \neq p_1$. В тот момент, когда следующая точка в выпуклой
оболочке совпала с первой, алгоритм останавливается — выпуклая оболочка
построена (рис. \ref{jarvis-simple-ex}).

\addimghere{pix/jarvis-match}{1}{Пример работы алгоритма Джарвиса.}
{jarvis-simple-ex}

Реализация алгоритма Джарвиса на языке Python находится в приложении В.
Также протестируем алгоритм Джарвиса на случайном наборе точек и
получим результат, представленный на на рисунке \ref{jarvis-ex}.

\addimghere{pix/jarvis_test}{1}{Сгенерированные точки (слева) и их 
выпуклая оболочка, найденная с помощью алгоритма Джарвиса (справа).}
{jarvis-ex}

Можно видеть, что алгоритм отработал верно, но у него есть один 
существенный недостаток -- он намного медленнее по сравнению с 
алгоритмом Грэхема.

\subsubsection{Алгоритм Киркпатрика}

Алгоритм Киркпатрика \cite{kirkpatrick} заключается в построении
выпуклой оболочки методом "разделяй и властвуй".

{\bf Описание алгоритма:}

Дано множество $S$, состоящее из $N$ точек. 
\begin{enumerate}
	\item Если $N \leq N_0$ ($N_0$ -- некоторое небольшое целое число),
то построить выпуклую оболочку одним из известных методов и
остановиться, иначе перейти к шагу 2.
	\item Разобьём исходное множество $S$ произвольным образом на два 
примерно равных по мощности подмножества $S_1$ и $S_2$ (пусть $S_1$
содержит $N/2$ точек, а $S_2$ содержит $N-N/2$ точек).
	\item Рекурсивно находим выпуклые оболочки каждого из подмножеств
$S_1$ и $S_2$.
	\item Строим выпуклую оболочку исходного множества как выпуклую 
оболочку объединения двух выпуклых многоугольников $CH(S_1)$ и
$CH(S_2)$.
\end{enumerate}
Поскольку $CH(S)=CH(S_1 \cup S_2) = CH(CH(S_1)\cup CH(S_2))$,
сложность этого алгоритма является решением рекурсивного соотношения
$T(N)\leq 2T(N/2)+f(N)$, где $f(N)$ -- время построения выпуклой
оболочки объединения двух выпуклых многоугольников, каждый из которых
имеет около $N/2$ вершин. Таким образом, временная сложность
этого алгоритма равна O($N\log{N}$).

Реализация алгоритма Киркпатрика на языке Python находится в
приложении Г. Результат работы показан на рисунке \ref{kirkpatrick-ex}.

\addimghere{pix/kirkpatrick_test}{1}{Сгенерированные точки (слева) и их 
выпуклая оболочка, найденная с помощью алгоритма Киркпатрика (справа).}
{kirkpatrick-ex}

\subsubsection{Сравнение алгоритмов}

Для того чтобы выбрать наиболее быстрый алгоритм, проведём их сравнение
по времени работы на 1000 тестовых случайно сгенерированных данных.

Результаты работы сведены в таблицу \ref{table-compare-hull}.

\begin{table}[h]
\begin{center}
\begin{tabular}{|l|c|c|c|}
\hline Алгоритм & Максимальное время, мс &
Минимальное время, мс & Среднее время, мс\\
\hline  Грэхема &  5317 & 581 & 2193.0 \\
\hline  Джарвиса & 187204 & 5979 & 66401.0 \\
\hline  Киркпатрика & 23102 & 2008  & 6944.0 \\
\hline
\end{tabular}
\end{center}
\caption{\label{table-compare-hull} Сравнение алгоритмов построения
выпуклой оболочки.}
\end{table}

Можно видеть, что самым эффективным в данном случае оказался алгоритм
Грэхема, именно его и будем использовать.

Алгоритм Грэхема уже реализован в библиотеке OpenCV в виде функции
{\tt convexHull()}.

Результат построения контура и выпуклой оболочки кисти руки
изображён на рисунке 
\ref{hull-example}.

\addimghere{pix/convex_hull_example}{0.5}{Выделение контура и
выпуклой его оболочки на кисти.}{hull-example}

Для определения необходимых ключевых точек на кисти необходимо 
найти {\it дефекты выпуклости} контура и выпуклой оболочки. 

Дефект выпуклости в данном случае -- это максимальное расстояние
между выпуклой оболочкой и контуром (рис. \ref{convex-defect-simple}).

\addimghere{pix/defects}{0.4}{Дефекты выпуклости.}
{convex-defect-simple}

В библиотеке OpenCV уже существует реализация расчётов дефектов 
выпуклости в виде функции {\tt convexityDefects()}. 

Результат расчёта дефектов выпуклости представлен на рисунке \ref{defects-example-1}.

\addimghere{pix/defects_example}{0.6}{Пример вычисления точек
дефектов выпуклости (красные).}{defects-example-1}

Можно видеть, что некоторые точки очень близко находятся друг к другу.
Это связано с тем, что выпуклая оболочка касается контура 
несколько раз в почти одном и том же месте. Чтобы избавиться от
этого напишем функцию очистки этих точек (приложение Д).

Функция работает следующим образом. Берётся первая попавшаяся точка
дефекта выпуклости, затем перебираются оставшиеся, пока они попадают
в круг радиусом {\tt ALPHA} с центром в первой точке. Эти точки не
попадают в итоговый результат. Затем берётся следующая точка и т.д.
Если после прохождения всех точек их количество больше 7 (2 точки
снизу руки + максимум 5 точек-"пальцев"), то {\tt ALPHA}
инкрементируется.

Результаты работы программы поиска дефектов выпуклости с 
очисткой "ненужных" точек выпуклости можно видеть на рисунке
\ref{defects-clean-ex}.

\addtwoimghere{pix/defects_clean_example}{pix/defects_clean_example1}
{Результат очистки дефектов выпуклости (оставшиеся точки 
отмечены на рисунках желтым цветом).}{defects-clean-ex}

Для того чтобы окончательно найти все точки на руке необходимо найти
центр масс контура. 

{\it Центром масс} фигуры является арифметическое среднее всех точек
фигуры. Пусть фигура состоит из $n$ отдельных точек 
$x_1, \ldots, x_n$, тогда центр масс определяется как
$$ c = \frac{1}{n}\sum_{i=1}^n{x_i}. $$

В контексте обработки изображения и компьютерного зрения каждая фигура
состоит из пикселей и центром масс является взвешенное среднее всех
пикселей, составляющих фигуру.

Центр масс контура можно найти с помощью {момента}. Момент изображения
-- это конкретное средневзвешенное значение интенсивности пикселей
изображения. Как и во всех остальных науках, моменты характеризуют
распределение материи относительно точки или оси. Формула для
момента изображения выглядит следующим образом:
\begin{equation}
M_{ij} = \sum_x{\sum_y{x^iy^iI(x,y)}}, 
\label{moments-equat}
\end{equation}
где $I(x,y)$ -- интенсивность пикселя в точке $(x,y)$,
$x$, $y$ относятся к строке и столбцу изображения. 

Проблема формулы (\ref{moments-equat}) состоит в том, что
моменты чувствительны к позициям $x$ и $y$. Если есть необходимость
в определении моментов, независимых к месту расположения контура, то
нужно использовать формулу {\it центральных моментов}:
$$M_{pq} = \sum_x{\sum_y{(x-\bar x)^p(y-\bar y)^qI(x,y)}}, $$
где $\bar x$ и $\bar y$ -- средние значения $x$ и $y$ соответственно.

Координата центра масс изображения таким образом будет определяться
как:

\begin{equation}
	\begin{aligned}
		C_x&=\frac{M_{10}}{M_{00}}, \\
		C_y&=\frac{M_{01}}{M_{00}},
	\end{aligned}
	\label{centroid-equat}
\end{equation}
где $(C_x, C_y)$ -- координата центра масс изображения.

В библиотеке OpenCV есть функция {\tt moments()}, определяющая
моменты изображения. Добавив в предыдущий код определение
координаты центра масс и построение лучей, получим результат, 
отображённый на рисунке \ref{keypoints-ex}.

\addtwoimghere{pix/defects_and_centroid_example}
{pix/defects_and_centroid_example1}{Результат поиска ключевых точек
руки.}{keypoints-ex}

\subsection{\nohyphens{Локализация ключевых точек кисти руки на изображении
на основе непрерывного скелета}}

Для локализации ключевых точек в статье \cite{nosov} предлагается
использовать непрерывный скелет. Предполагается, что уже успешно 
был выполнен этап сегментации и имеется отсегментированное
изображение с силуэтом кисти руки. На основе контурного представления
силуэта жеста строится его скелет. Для определения скелета
используется понятие {\it максимального пустого круга}.

{\bf Определение 1}. Для многоугольной фигуры $F$ {\it максимальным
пустым кругом} называется всякий круг $B$, полностью 
содержащийся внутри фигуры $F$, такой, что любой другой круг $B'$,
содержащийся внутри фигуры $F$, не содержит в себе $B$.

{\bf Определение 2}. {\it Скелетом} многоугольной фигуры $F$
является множество центров её максимальных пустых кругов.

Непрерывный скелет многоугольной фигуры является подмножеством
диаграммы Воронова \cite{mesteckij}. Совокупность общих линий всех пар 
несмежных ячеек диаграммы Воронова образуют ветви скелета 
\cite{mesteckij}. На скелете определена радиальная функция $R(x,y)$,
ставящая в соответствие каждой точке скелета $(x,y)$ значение
радиуса максимального пустого круга с центром в этой точке.

В большинстве случаев скелет ладони имеет шумы в виде
малозначимых ветвей, которые как правило, мешают дальнейшему анализу.
Для удаления шумовых ветвей используется дополнительная обработка,
называемая "стрижкой" \cite{mesteckij}. Процесс "стрижки" заключается
в удалении ветвей, граничащих с контурами силуэта руки.

Существующие эффективные алгоритмы позволяют выполнять построение 
скелета за время $O(N\log{N})$, где $N$ -- число вершин в
многоугольнике \cite{mesteckij-rejer}. В связи с тем, что скорость
построения скелета напрямую зависит от количества углов
многоугольной фигуры, то для ускорения построения скелета можно
применить аппроксимацию этой фигуры \cite{nosov-2}. 

Демонстрация процесс построения скелета представлена на рисунке
\ref{skeleton-ex}. На основе непрерывного скелета и радиальной
функции $R(x,y)$ можно с большой точностью вычислить координаты
кончиков пальцев и координаты центра ладони. Каждый палец может
принимать два условных состояния: сжатый в кулак или разжатый.
Все ветви скелета, соответствующие пальцу, оканчиваются вершиной
степени 1. Ветвь пальца можно разделить на две части: палец и пясть.

\addimghere{pix/skeleton_example}{1}{Процесс построения скелета:
{\it a} -- исходное изображение; {\it б} -- аппроксимированное
изображение; {\it в} -- скелет многоугольника; {\it г} -- 
скелет после стрижки}{skeleton-ex}

Для классификации ветвей пальцев используется набор эвристических 
правил:
\begin{enumerate}
	\item Ветвь пальца лежит на графе между вершинами со степенями
1 и 3.
	\item Радиальная функция ветви на вершине степени 1 увеличивается
более чем в 2,5 раза по сравнению с вершиной степени 3.
	\item Радиальная функция начинает резко расти, то есть частные
производные $R'$ больше заданного порога.
\end{enumerate}

Первая точка на ветви, где производная радиальной функции превышает
заданный порог, является точкой конца пальца. Центром ладони будем
считать точку, лежащую на скелете ладони, радиальная функция
которой принимает максимальное значение. На рисунке 
\ref{skeleton-result-ex} демонстрируется результат вычисления
ключевых точек на изображении.

Для распознавания простого, ограниченного набора жестов достаточно
составить набор эвристических правил, основанных на следующих
данных: количество пальцев, их длина, количество циклов в графе и их 
габариты. В более сложных случаях набора эвристических правил мало, и
для распознавания жестов применяются дескрипторы формы кисти руки,
состоящие из определённых инвариантных признаков.

Достоинством метода распознавания жестов на основе непрерывного
скелета является его быстродействие, высокая точность локализации
особых точек и, как следствие, высокий результат распознавания. 

\addtwoimghere{pix/skeleton_result_1}{pix/skeleton_result_2}
{Определение ключевых точек: {\it а} -- исходное изображение;
{\it б} -- скелет и ключевые точки на нём.}{skeleton-result-ex}


Эффект рождения электрон-позитронных пар из вакуума под действием
электрического поля впервые обсуждался, по-видимому, в работах
Заутера \cite{sauter} в связи с так называемым парадоксом Клейна
(см., например, \cite{zommerfeld}). Вероятность перехода
вакуум-вакуум, которая в присутствии постоянного однородного
электромагнитного поля отлична от единицы за счёт эффекта рождения
$e^+ e^-$-пар, в главном приближении была найдена Гейзенбергом и
Эйлером \cite{heisenberg}, точные формулы в случаях вакуума
заряженных частиц со спином 0 и 1/2 получены Швингером
\cite{schwinger}, а в случае векторных бозонов --- Ваняшиным и
Терентьевым \cite{vanyashin}. Вероятность рождения $e^+ e^-$-пар из
вакуума становится заметно отличной от нуля при напряжённости
постоянного электрического поля, близкой к характерному для
квантовой электродинамики (КЭД) значению
$$E_S=\frac{m^2c^3}{e\hbar}=1.32\cdot 10^{16}\,\,В/см $$                % Формула посередине без нумерации
(см. \cite{sauter,zommerfeld,heisenberg,schwinger}), при котором
электрическое поле на комптоновской длине
 $$l_C =\frac{\hbar}{mc}=3.86\cdot 10^{-11}\,\,см$$
совершает над электроном работу $mc^2$. Постоянное поле такой
напряжённости вряд ли может быть получено в лабораторных условиях.
Поэтому многие авторы сосредоточились на теоретическом исследовании
процесса рождения пар под действием переменных во времени
электрических полей
\cite{bunkin,brezin,popov1,popov2,narozhny1,mostepanenko,marinov,grib,ringwald,popov3},
хотя и в этом случае надежды на достижение напряжённостей порядка
$E_S$ до последнего времени казались весьма призрачными.

\begin{center}
$I \sim I_S=(c/4\pi)E_S ^2=4.65\cdot 10^{29}$ Вт/см$^2$.                % Другой способ записать формулу посередине без нумерации
\end{center}

............................................................

Однако существуют процессы, для описания которых модель плоской
волны использовать невозможно. В частности, плоская электромагнитная
волна произвольной интенсивности и спектрального состава не рождает
$e^+ e^-$-пар из вакуума \cite{schwinger}, поскольку оба инварианта
электромагнитного поля плоской волны
\[\mathcal F=(\mathbf E^2-\mathbf H^2)/2,\quad\mathcal G=(\mathbf E\cdot\mathbf H)\]    % Третий способ -"-"- и пример каллиграфических букв
равны нулю. Поэтому в настоящей работе для описания
электромагнитного поля фокусированной волны использована
реалистическая трёхмерная модель, предложенная в работе
\cite{narozhny2}. Эта модель основана на точном решении уравнений
Максвелла и была успешно применена в работе \cite{narozhny3} для
количественного объяснения эффекта анизотропии углового
распределения электронов, ускоренных интенсивным лазерным импульсом,
который наблюдался в эксперименте \cite{malka}. Сразу же отметим,
что использование суперпозиции двух фокусированных импульсов
позволяет обнаружить рождение $e^+ e^-$-пар при интенсивностях,
значительно меньших, чем в случае одиночного импульса
\cite{narozhny4,narozhny5}.

\begin{equation}                % Формула с нумерацией и лейблом "pairs" для ссылки на неё с помощью (\ref{pairs})
 \label{pairs}
 \begin{aligned}                % Нужно для набора многострочных формул; в отличие от array, сохраняет размер символов в дробях и кое-что ещё...
  N=\frac{e^2E_S^2}{4\pi^2\hbar^2c}\int\limits_{V}dV\int\limits_{0}^{\tau}dt\,
  \epsilon\eta\, \mathrm{cth}\,
  \frac{\pi\eta}{\epsilon}\exp\left(-\frac{\pi}{\epsilon}\right).
 \end{aligned}
\end{equation}
Здесь
$$\epsilon=\mathcal E/E_S,\quad\eta=\mathcal H/E_S$$
--- приведённые поля, а $\mathcal E$ и $\mathcal H$ --- инварианты,
имеющие смысл напряжённостей электрического и магнитного поля в той
системе отсчёта, где они параллельны:

\newpage
\section{Раздел 2}

Текст раздела.

................................................

\begin{equation}
 \begin{aligned}
  \mathbf E^h=\pm i\mathbf H^e,\quad\mathbf H^h=\mp i\mathbf E^e. % Пример жирного шрифта
 \end{aligned}
\end{equation}

.................................................

\begin{equation}            % Пример большой формулы, где нужно переносить часть выражения на другую строчку
 \label{Eeexplicit}         % Когда нужны большие скобки, их можно открывать и закрывать с помощью \left( и \right(
 \begin{aligned}            % для случая круглых скобок. Когда надо открыть скобку на одной строке, а закрыть на другой,
  \mathbf E^e=\frac{E_0e^{-i\varphi}}{(1+2i\chi)^2}\exp\left(-\frac{\xi^2}      % надо в конце первой сроки поставить \right.,
  {1+2i\chi}\right)\left\{\left(1-\frac{\xi^2}{1+2i\chi}\right)\mathbf          % а в начале следующей - \left.
  e_x+\right.\\
  \left.+\frac{\xi^2}{1+2i\chi}(\cos2\phi\,\mathbf e_x+\sin2\phi\,\mathbf e_y)\right\}
 \end{aligned}
\end{equation}
\begin{equation}
 \label{Heexplicit}
 \begin{aligned}
  \mathbf H^e=\frac{E_0e^{-i\varphi}}{(1+2i\chi)^2}\exp\left(-\frac{\xi^2}
  {1+2i\chi}\right)\left\{\left[1-\frac{\xi^2}{1+2i\chi}-\right.\right.\\
  \left.-\frac{2\Delta^2}{1+2i\chi}\left(2-\frac{4\xi^2}
  {1+2i\chi}+\frac{\xi^4}{(1+2i\chi)^2}\right)\right]\mathbf e_y-\\
  -\frac{\xi^2}{1+2i\chi}\left[1-\frac{2\Delta^2}{1+2i\chi}\left(3-\frac{\xi^2}{1+2i\chi}\right)\right]
  (\sin2\phi\,\mathbf e_x-\cos2\phi\,\mathbf e_y)-\\
  \left.-\frac{4i\Delta\xi}{1+2i\chi}
  \left(2-\frac{\xi^2}{1+2i\chi}\right)\sin\phi\,\mathbf e_z\right\}
 \end{aligned}
\end{equation}


\newpage
\section{Раздел 3}

Текст раздела


\subsection{Подраздел 1}

        % Пример таблицы. \multirow{x}{Ycm} позволяет объединить x строк в столбце, длину которого задаём равной Y см

\begin{table}[t]
\caption{\label{table1}Среднее число пар, рождённых одиночным
(слева) и двумя сталкивающимися (справа) циркулярно-поляризованными
импульсами $e$-типа из вакуума, $\Delta=0.1$}
\begin{center}
\begin{tabular}{|c|c|c|c|c|c|}
\hline \multirow{2}{2cm}{$I\cdot10^{-28}$,
Вт/см$^2$}&\multirow{2}{2cm}{\quad$E_0/E_S$}&
\multirow{2}{2.5cm}{$\qquad\,
N$}&\multirow{2}{2.5cm}{$I\cdot10^{-26}$, Вт/см$^2$}&
\multirow{2}{2.5cm}{\quad$E_0/E_S$}&\multirow{2}{2.5cm}{$\qquad\, N$}\\
&&&&&\\
\hline   0.6   &  0.203 &    1.94(-5) &1.0     &0.0262   & 2.36(-8)\\
\hline   0.8   &  0.234 &    5.57(-2) &1.5     &0.0321   & 3.12(-3)\\
\hline   1.0   &  0.262 &       13.4  &2.0     &0.0371   &    3.85\\
\hline   1.5   &  0.321 &     7.57(4) &2.5     &0.0414   &  5.20(2)\\
\hline   2.0   &  0.371 &     1.42(7) &3.0     &0.0454   &  2.01(4)\\
\hline   2.5   &  0.414 &     5.29(8) &4.0     &0.0524   &  3.59(6)\\
\hline   3.0   &  0.454 &     7.89(9) &5.0     &0.0586   &  1.33(8)\\
\hline   4.0   &  0.524 &    3.70(11) &6.0     &0.0642   &  1.95(9)\\
\hline   5.0   &  0.586 &    5.35(12) &7.0     &0.0693   & 1.61(10)\\
\hline   6.0   &  0.642 &    4.05(13) &8.0     &0.0741   & 8.94(10)\\
\hline   8.0   &  0.741 &    7.17(14) &9.0     &0.0786   & 3.75(11)\\
\hline  10.0   &  0.829 &    5.33(15) &10.0    &0.0829   & 1.28(12)\\
\hline
\end{tabular}
\end{center}
\end{table}

\subsection{Подраздел 2}

Текст подраздела

\newpage
\section{Заключение}

Текст заключения

Рассмотренные случаи циркулярно-поляризованной и
линейно-поляризованной волн позволяют сделать утверждение, что для
процесса рождения $e^+e^-$-пар эффективнее сталкивать лазерные
импульсы, а не использовать одиночные, поскольку в этом случае порог
рождения пар составляет величину $I \sim 10^{26}$ Вт/см$^2$, что на
один-два порядка меньше, чем в случае одиночного импульса. Более
того, в данной работе впервые показано, что эффективнее использовать
лазерные импульсы с линейной поляризацией. Это вызвано в первую
очередь тем, что при одной и той же интенсивности лазерного импульса
напряжённость электрического поля линейно-поляризованной волны в
$\sqrt{2}$ больше, чем у волны с циркулярной поляризацией. Как
показывают расчёты, логарифм числа частиц как функция $E_0$ -
напряжённости каждого поля по отдельности, - практически одинаков
для обеих поляризаций. Но если рассматривать $\lg N$ как функцию
интенсивности лазерного импульса, то получится, что поле с линейной
поляризацией рождает на несколько порядков пар больше, что довольно
существенно. Это означает, что на эксперименте эффективнее будет
использовать именно линейно-поляризованное электромагнитное поле.

\newpage
\begin{thebibliography}{99}                    % Список литературы
\addcontentsline{toc}{section}{Список литературы}

\bibitem{dix}
Dix A., Finlay J., Abowd G.D., Beale R. Human-Computer Interaction. - Third Edition, Pearson
Education Limited: 2004. - p. 857

\bibitem{jiangqin}
Jiangqin W., Wen G. A FAst Sign Word Recognition Method for Chinese Sign Language // In 
Proceedings of the Third International Conference on Advances in Multimodal Interfaces (ICMI 
'00). - Springer-Verlag, London, 2000. - p.599-606

\bibitem{sanna}
Sanna A., Lamberti F., Paravati G., Henao R., Eduardo A., Manuri F. A Kinect-Based Natural
Interface for Quadrotor Control // Intelligent Technologies for Interactive Entertainment, 
Volume 78. Springer Berlin Heidelberg, 2012. - p. 48-56


\bibitem{ycbcr-bib}
Basilio, Jorge \& Torres, Gualberto \& Sanchez-Perez, Gabriel \& Toscano,
Karina \& Perez-Meana, Hector. Explicit image detection using
YCbCr space color model as skin detection, 2011. - p. 123-128. 

\bibitem{otsu}
N. Otsu. A threshold selection method from gray-level histograms
(англ.) // IEEE Trans. Sys., Man., Cyber. : journal. — 1979. — Vol. 9.
— p. 62—66.

\bibitem{van-vibe}
M. Van Droogenbroeck, O. Barnich. Vibe: A disruptive
method for background subtraction. // In T. Bouwmans, F.
Porikli, B. Hoferlin, A. Vacavant, editors, Background
Modeling and Foreground Detection for Video Surveillance,
chapter 7. Chapman and Hall/CRC, pages 7.1-7.23, July
2014.

\bibitem{MOG-1}
P. Kaewtrakulpong, R. Bowden. An improved adaptive 
background mixture model for real-time tracking with
shadow detection // Video-Based Surveillance Systems, pp.
135-144. Springer, 2002.

\bibitem{MOG-2}
Z. Zivkovic, F. Heijden. Efficient adaptive density
estimation per image pixel for the task of background
subtraction // Pattern recognition letters, Vol. 27(7),
pp. 773-780, 2006.

\bibitem{MOG-3}
Z. Zivkovic, F. Heijden. Efficient adaptive density
estimation per image pixel for the task of background
subtraction // Pattern recognition letters, Vol. 27(7),
pp. 773-780, 2006.

\bibitem{satoshi}
Satoshi S., Keiich A. 1985. Topological Structural Analysis of
Digitized Binary Images by Border Following. Computer vision, graphics,
and image processing, 30. 

\bibitem{ten-chin}
C-H Teh and Roland T. Chin. On the detection of dominant points on
digital curves. Pattern Analysis and Machine Intelligence, IEEE
Transactions on, 11(8):859–872, 1989.

\bibitem{graham}
Graham R. L. An efficient algorithm for determining the
convex hull of a finite planar set //Info. Pro. Lett. –
1972. – Т. 1. – С. 132-133.

\bibitem{jarvis}
R.A. Jarvis. On the identification of the convex hull of a finite set of
points in the plane // Information Processing Letters, Volume 2,
Issue 1, 1973, p. 18-21.

\bibitem{kirkpatrick}
Kirkpatrick, David G.; Seidel, Raimund (1986). "The ultimate planar
convex hull algorithm?". SIAM Journal on Computing. 15 (1): 287–299.

\bibitem{nosov}
А.В. Носов. Локализация ключевых точек кисти руки на изображении на
основе непрерывного скелета. Математические методы моделирования,
управления и анализа данных, с. 77-79, 2015.

\bibitem{mesteckij}
Местецкий Л. М. Непрерывная морфология бинарных изображений: фигуры,
скелеты, циркуляры. М. : Физматлит, 2009.

\bibitem{mesteckij-rejer}
Местецкий Л. М., Рейер И. Непрерывное скелетное представление 
изображения с контролируемой точностью // International Conference
Graphicon. M., 2003. С. 51–54.

\bibitem{nosov-2}
Носов А. В. Алгоритм распознавания жестов
рук на основе скелетной модели кисти руки // Вестник
СибГАУ. 2014. Вып. 2(54). С. 62–67.


\end{thebibliography}



\newpage

\anonsection{ПРИЛОЖЕНИЯ}

\subsection*{Приложение А. Код реализации алгоритма Оцу и построения изображений.}
\addcontentsline{toc}{subsection}{Приложение А}

\begin{minted}[mathescape, 
				linenos,
				gobble=0, 
				frame=lines,
				framesep=2mm]{python}
import numpy as np
import cv2
import matplotlib.pyplot as plt

INPUT_TO_OUTPUT = {'image3.jpg':
                       {'histogram': 'histogram_for_3.png',
                        'output': 'otsu_exmpl_for_3.png'},
                   'image7.jpg':
                       {'histogram': 'histogram_for_7.png',
                        'output': 'otsu_exmpl_for_7.png'}}
def get_min_max(image_flatten: np.ndarray) -> tuple:
    min_pix, max_pix = image_flatten[0], image_flatten[0]
    for pixel in image_flatten:
        if pixel < min_pix:
            min_pix = pixel
        if pixel > max_pix:
            max_pix = pixel
    return min_pix, max_pix
def create_histogram(image_flatten: np.ndarray, 
					size: int, min_pix: int) -> np.array:
    hist = np.zeros(size)
    for pixel in image_flatten:
        hist[pixel - min_pix] += 1
    return hist
def otsu_threshold(image: np.ndarray):
    img_gray = cv2.cvtColor(image, cv2.COLOR_BGR2GRAY)
    img_gray_flatten = img_gray.flatten()
    min_pixel, max_pixel = get_min_max(img_gray_flatten)
    size = max_pixel - min_pixel + 1
    histogram = create_histogram(img_gray_flatten, size, min_pixel)
    m = 0
    n = 0
    for i in range(size):
        m += i * histogram[i]
        n += histogram[i]
    max_sigma = -1
    threshold = 0
    alpha1 = 0
    beta1 = 0
    for i in range(size - 1):
        alpha1 += i * histogram[i]
        beta1 += histogram[i]
        w1 = float(beta1) / n  # Вероятность класса 1
        w2 = 1 - w1
        a = float(alpha1) / beta1 - float(m - alpha1) / (n - beta1) 
        sigma = w1 * w2 * a * a
        if sigma > max_sigma:
            max_sigma = sigma
            threshold = i
    threshold += min_pixel
    return threshold
def save_histogram(image, fname: str):
    image = cv2.cvtColor(image, cv2.COLOR_BGR2GRAY).flatten()
    plt.hist(image, 255)
    plt.xlim([0, 255])
    plt.savefig(fname)
    plt.show()
if __name__ == '__main__':
    for fname, outputs in INPUT_TO_OUTPUT.items():
        image = cv2.imread(fname)
        save_histogram(image, outputs['histogram'])
        thresh = otsu_threshold(image)
        print(f'Threshold = {thresh}')
        ret, img_thresh = cv2.threshold(cv2.cvtColor(image, cv2.COLOR_BGR2GRAY), 
        	thresh, 255, cv2.THRESH_BINARY)
        imask = img_thresh == 255
        canvas = np.zeros_like(image, np.uint8)
        canvas[imask] = image[imask]
        cv2.imwrite(outputs['output'], canvas)
\end{minted}

\newpage

\subsection*{Приложение Б. Код реализации алгоритма Грэхема.}
\addcontentsline{toc}{subsection}{Приложение Б}

\begin{minted}[mathescape, 
				linenos,
				gobble=0, 
				frame=lines,
				framesep=2mm]{python}
from functools import cmp_to_key

class Point:
    def __init__(self, x=None, y=None):
        self.x = x
        self.y = y

# Расстояние между двумя точками
def dist_sq(p1, p2):
    return ((p1.x - p2.x) * (p1.x - p2.x) +
            (p1.y - p2.y) * (p1.y - p2.y))

# Определение ориентации трёх точек
# (перекрестное произведение векторов pq и qr)
def orientation(p, q, r):
    val = ((q.y - p.y) * (r.x - q.x) -
           (q.x - p.x) * (r.y - q.y))
    if val == 0:
        return 0  # коллинеарны
    elif val > 0:
        return 1  # по часовой
    else:
        return 2  # против часовой

# Сравнение двух точек для сортировки
def compare(p1, p2):
    global p0
    o = orientation(p0, p1, p2)
    if o == 0:
        return -1 if dist_sq(p0, p2) >= dist_sq(p0, p1) else 1
    else:
        return -1 if o == 2 else 1

def convex_hull(input_points: list) -> list:
    # Конвертируем в класс Point
    points = [Point(point[0], point[1]) for point in input_points]
    n = len(input_points)
    # Находим минимальную точку
    min_y = points[0].y
    min_i = 0
    for i in range(1, n):
        y = points[i].y
        if ((y < min_y) or
                (min_y == y and points[i].x < points[min_i].x)):
            min_y = points[i].y
            min_i = i

    points[0], points[min_i] = points[min_i], points[0]

    # Сортируем массив
    global p0
    p0 = points[0]
    points = sorted(points, key=cmp_to_key(compare))

    m = 1  # Начальный размер нового массива
    # Удаляем лишние точки
    for i in range(1, n):
        while ((i < n - 1) and
               (orientation(p0, points[i], points[i + 1]) == 0)):
            i += 1
        points[m] = points[i]
        m += 1
    if m < 3:
        return None

    S = [points[0], points[1], points[2]]
    for i in range(3, m):
        while ((len(S) > 1) and
               (orientation(S[-2], S[-1], points[i]) != 2)):
            S.pop()
        S.append(points[i])

    return [(p.x, p.y) for p in S]
\end{minted}

\newpage

\subsection*{Приложение В. Код реализации алгоритма Джарвиса.}
\addcontentsline{toc}{subsection}{Приложение В}

\begin{minted}[mathescape, 
				linenos,
				gobble=0, 
				frame=lines,
				framesep=2mm]{python}
class Point:
    def __init__(self, x, y):
        self.x = x
        self.y = y

# поиск самой левой нижней точки
def left_index(points):
    minn = 0
    for i in range(1, len(points)):
        if points[i].x < points[minn].x or \
                (points[i].x == points[minn].x and
                 points[i].y > points[minn].y):
            minn = i
    return minn

# аналогично orientation в алгоритме Грэхема
def orientation(p, q, r):
    val = (q.y - p.y) * (r.x - q.x) - \
          (q.x - p.x) * (r.y - q.y)
    if val == 0:
        return 0
    elif val > 0:
        return 1
    return -1

def convex_hull(init_points):
    n = len(init_points)
    if n < 3:
        return
    points = [Point(p[0], p[1]) for p in init_points]
    l = left_index(points)
    hull = []
    p = l
    q = 0
    while True:
        hull.append(p)
        q = (p + 1) % n
        for i in range(n):
            if (orientation(points[p],
                            points[i], points[q]) == 2):
                q = i
        p = q
        if p == l:
            break
    return [(points[i].x, points[i].y) for i in hull]
\end{minted}

\newpage

\subsection*{Приложение Г. Код реализации алгоритма Киркпатрика.}
\addcontentsline{toc}{subsection}{Приложение Г}

\begin{minted}[mathescape, 
				linenos,
				gobble=0, 
				frame=lines,
				framesep=2mm]{python}
from collections import namedtuple
from random import randint

Point = namedtuple('Point', 'x y')

def flipped(points):
    return [Point(-point.x, -point.y) for point in points]

def quickselect(ls, index, lo=0, hi=None, depth=0):
    if hi is None:
        hi = len(ls) - 1
    if lo == hi:
        return ls[lo]
    if len(ls) == 0:
        return 0
    pivot = randint(lo, hi)
    ls = list(ls)
    ls[lo], ls[pivot] = ls[pivot], ls[lo]
    cur = lo
    for run in range(lo+1, hi+1):
        if ls[run] < ls[lo]:
            cur += 1
            ls[cur], ls[run] = ls[run], ls[cur]
    ls[cur], ls[lo] = ls[lo], ls[cur]
    if index < cur:
        return quickselect(ls, index, lo, cur-1, depth+1)
    elif index > cur:
        return quickselect(ls, index, cur+1, hi, depth+1)
    else:
        return ls[cur]

def bridge(points, vertical_line):
    candidates = set()
    if len(points) == 2:
        return tuple(sorted(points))
    pairs = []
    modify_s = set(points)
    while len(modify_s) >= 2:
        pairs += [tuple(sorted([modify_s.pop(), modify_s.pop()]))]
    if len(modify_s) == 1:
        candidates.add(modify_s.pop())
    slopes = []
    for pi, pj in pairs[:]:
        if pi.x == pj.x:
            pairs.remove((pi, pj))
            candidates.add(pi if pi.y > pj.y else pj)
        else:
            slopes += [(pi.y-pj.y)/(pi.x-pj.x)]
    median_index = len(slopes)//2 - (1 if len(slopes) % 2 == 0 else 0)
    median_slope = quickselect(slopes, median_index)
    small = {pairs[i] for i, slope in enumerate(slopes) if slope < median_slope}
    equal = {pairs[i] for i, slope in enumerate(slopes) if slope == median_slope}
    large = {pairs[i] for i, slope in enumerate(slopes) if slope > median_slope}
    max_slope = max(point.y-median_slope*point.x for point in points)
    max_set = [point for point in points if \
    	point.y-median_slope*point.x == max_slope]
    left = min(max_set)
    right = max(max_set)
    if left.x <= vertical_line < right.x:
        return left, right
    if right.x <= vertical_line:
        candidates |= {point for _, point in large | equal}
        candidates |= {point for pair in small for point in pair}
    if left.x > vertical_line:
        candidates |= {point for point, _ in small | equal}
        candidates |= {point for pair in large for point in pair}
    return bridge(candidates, vertical_line)

def connect(lower, upper, points):
    if lower == upper:
        return [lower]
    max_left = quickselect(points, len(points)//2-1)
    min_right = quickselect(points, len(points)//2)
    left, right = bridge(points, (max_left.x + min_right.x)/2)
    points_left = {left} | {point for point in points if point.x < left.x}
    points_right = {right} | {point for point in points if point.x > right.x}
    return connect(lower, left, points_left) + connect(right, upper, points_right)

def upper_hull(points):
    lower = min(points)
    lower = max({point for point in points if point.x == lower.x})
    upper = max(points)
    points = {lower, upper} | {p for p in points if lower.x < p.x < upper.x}
    return connect(lower, upper, points)

def convex_hull(init_points):
    points = [Point(p[0], p[1]) for p in init_points]
    upper = upper_hull(points)
    lower = flipped(upper_hull(flipped(points)))
    if upper[-1] == lower[0]:
        del upper[-1]
    if upper[0] == lower[-1]:
        del lower[-1]
    return upper + lower
\end{minted}

\newpage

\subsection*{Приложение Д. Код функции очистки лишних точек дефектов выпуклости.}
\addcontentsline{toc}{subsection}{Приложение Д}

\begin{minted}[mathescape, 
				linenos,
				gobble=0, 
				frame=lines,
				framesep=2mm]{python}
import math
import numpy as np

def squeeze(array):
    array = array[0] if len(array) == 1 else array
    return [p[0] for p in array] if len(array[0]) == 1 else array

# "очистка" выпуклой оболочки до не больше 7 точек
def clear_convex_hull(hull_index, contour):
    distance_point = lambda p1, p2: \
    	math.sqrt((p1[0] - p2[0]) ** 2 + (p1[1] - p2[1]) ** 2)
    hull_index, contour = squeeze(hull_index), squeeze(contour)
    ALPHA = 10
    while True:
        boundary = len(hull_index) - 1
        clean_hull, i = [], 0
        while i < boundary:
            clean_hull.append(hull_index[i])
            while i < boundary and \
                    distance_point(contour[hull_index[i]],
                    contour[hull_index[i + 1]]) < ALPHA:
                i += 1
            i += 1
        if len(clean_hull) > 7:
            ALPHA += 1
        else:
            break
    return np.array(clean_hull[:-1]) if \
        distance_point(contour[clean_hull[0]],
        contour[clean_hull[-1]]) < ALPHA else np.array(clean_hull)
\end{minted}


\end{document}
