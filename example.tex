\documentclass [a4paper,oneside,final,14pt]{extarticle}
%%%% Преамбула %%%
%
%\usepackage{fontspec} % XeTeX
%\usepackage{xunicode} % Unicode для XeTeX
%\usepackage{xltxtra}  % Верхние и нижние индексы
%\usepackage{pdfpages} % Вставка PDF
%
\usepackage[usenames,dvipsnames]{color}
\usepackage{geometry}           % пакет для задания полей страницы командой \geometry
\geometry{left=3cm,right=1.5cm,top=2cm,bottom=2cm}
\usepackage[utf8]{inputenc}   % кодировка текста
\usepackage{mathtext}           % позволяет использовать русские буквы в формулах
\usepackage[T2A]{fontenc}       %пакет Т2А необходим для правильного отображения кириллицы и переноса слов
%\inputencoding{cp1251}          % тоже кодировка...
\usepackage[russian]{babel}     % языковой пакет - последний язык главный
\usepackage[unicode]{hyperref}  %создаёт гиперссылки на список литературы в pdf-файле
\usepackage{amstext,amsmath,amssymb}            % пакеты для формул
\usepackage{bm}                 % boldmath - пакет для жирного шрифта
\usepackage[pdftex]{graphicx}   % пакет для включения рисунков в форматах png,pdf,jpg,mps,tif
\usepackage{titlesec}
\usepackage[title,titletoc]{appendix}

\usepackage{amsfonts}           % греческие символы и, возможно, что-то ещё
\usepackage{indentfirst}        % одинаковый отступ для первого параграфа и всего остального
\usepackage{cite}               % команда /cite{1,2,7,9} даёт ссылки
\usepackage{multirow}           % пакет для объединения строк в таблице: надо указать число строк и ширину столбца
\usepackage{array}              % нужен для создания таблиц
\linespread{1.3}                % полтора интервала. Если 1.6, то два интервала
\pagestyle{plain}               % номерует страницы

\usepackage[title,titletoc]{appendix}
\usepackage{listings} % Оформление исходного кода
\lstset{
	language=Python,
    basicstyle=\small\ttfamily, % Размер и тип шрифта
    breaklines=true, % Перенос строк
    tabsize=2, % Размер табуляции
    literate={--}{{-{}-}}2 % Корректно отображать двойной дефис
}
%
%% Шрифты, xelatex
%\defaultfontfeatures{Ligatures=TeX}
%\setmainfont{Times New Roman} % Нормоконтроллеры хотят именно его
%\newfontfamily\cyrillicfont{Times New Roman}
%%\setsansfont{Liberation Sans} % Тут я его не использую, но если пригодится
%\setmonofont{FreeMono} % Моноширинный шрифт для оформления кода
%
%% Русский язык
%\usepackage{polyglossia}
%\usepackage{amssymb,amsfonts,amsmath} % Математика
%\numberwithin{equation}{section} % Формула вида секция.номер
%
%\usepackage{enumerate} % Тонкая настройка списков
%\usepackage{indentfirst} % Красная строка после заголовка
%\usepackage{float} % Расширенное управление плавающими объектами
%\usepackage{multirow} % Сложные таблицы
%
%% Пути к каталогам с изображениями
%\usepackage{graphicx} % Вставка картинок и дополнений
%\graphicspath{{images/}{images/userguide/}{images/testing/}{images/infrastructure/}{extra/}{extra/drafts/}}
%
%% Формат подрисуночных записей
%\usepackage{chngcntr}
%\counterwithin{figure}{section}
%
%% Гиперссылки
\usepackage{hyperref}
\hypersetup{
    colorlinks, urlcolor={black}, % Все ссылки черного цвета, кликабельные
    linkcolor={black}, citecolor={black}, filecolor={black},
    pdfauthor={Дмитрий Юдаков},
    pdftitle={Исследование методов и разработка алгоритма детектирования, описания и извлечения конфигураций кисти человека}
}
%
%% Оформление библиографии и подрисуночных записей через точку
%\makeatletter
%\renewcommand*{\@biblabel}[1]{\hfill#1.}
%\renewcommand*\l@section{\@dottedtocline{1}{1em}{1em}}
%\renewcommand{\thefigure}{\thesection.\arabic{figure}} % Формат рисунка секция.номер
%\renewcommand{\thetable}{\thesection.\arabic{table}} % Формат таблицы секция.номер
%\def\redeflsection{\def\l@section{\@dottedtocline{1}{0em}{10em}}}
%\makeatother
%
%\renewcommand{\baselinestretch}{1.4} % Полуторный межстрочный интервал
%\parindent 1.27cm % Абзацный отступ
%
%\sloppy             % Избавляемся от переполнений
%\hyphenpenalty=1000 % Частота переносов
%\clubpenalty=10000  % Запрещаем разрыв страницы после первой строки абзаца
%\widowpenalty=10000 % Запрещаем разрыв страницы после последней строки абзаца
%
%% Отступы у страниц
%\usepackage{geometry}
%\geometry{left=3cm}
%\geometry{right=1cm}
%\geometry{top=2cm}
%\geometry{bottom=2cm}
%
%% Списки
\usepackage{enumitem}
\setlist[enumerate,itemize]{leftmargin=12.7mm} % Отступы в списках
%
%\makeatletter
%    \AddEnumerateCounter{\asbuk}{\@asbuk}{м)}
%\makeatother
\setlist{nolistsep} % Нет отступов между пунктами списка
%\renewcommand{\labelitemi}{--} % Маркет списка --
%\renewcommand{\labelenumi}{\asbuk{enumi})} % Список второго уровня
%\renewcommand{\labelenumii}{\arabic{enumii})} % Список третьего уровня
%
%% Содержание
%\usepackage{tocloft}
%\renewcommand{\cfttoctitlefont}{\hspace{0.38\textwidth}\MakeTextUppercase} % СОДЕРЖАНИЕ
%\renewcommand{\cftsecfont}{\hspace{0pt}}            % Имена секций в содержании не жирным шрифтом
%\renewcommand\cftsecleader{\cftdotfill{\cftdotsep}} % Точки для секций в содержании
%\renewcommand\cftsecpagefont{\mdseries}             % Номера страниц не жирные
%\setcounter{tocdepth}{3}                            % Глубина оглавления, до subsubsection
%
%%% Нумерация страниц справа сверху
%%\usepackage{fancyhdr}
%%\pagestyle{fancy}
%%\fancyhf{}
%%\fancyhead[R]{\textrm{\thepage}}
%%\fancyheadoffset{0mm}
%%\fancyfootoffset{0mm}
%%\setlength{\headheight}{17pt}
%%\renewcommand{\headrulewidth}{0pt}
%%\renewcommand{\footrulewidth}{0pt}
%%\fancypagestyle{plain}{ 
%%    \fancyhf{}
%%    \rhead{\thepage}
%%}
%
%% Формат подрисуночных надписей
\RequirePackage{caption}
\DeclareCaptionLabelSeparator{defffis}{. } % Разделитель
\captionsetup[figure]{justification=centering, labelsep=defffis, format=plain} % Подпись рисунка по центру
\captionsetup[table]{justification=centering, labelsep=defffis, format=plain, singlelinecheck=false} % Подпись таблицы слева
\addto\captionsrussian{\renewcommand{\figurename}{Рис.}} % Имя фигуры

%% Пользовательские функции


%\newcommand{\addimgapp}[2]{ % Это костыль для приложения Б
%    \begin{figure}[H]
%        \centering
%        \includegraphics[width=1\linewidth]{#1}
%        \caption*{#2}
%    \end{figure}
%}


\newcommand{\addimg}[4]{ % Добавление одного рисунка
    \begin{figure}
        \centering
        \includegraphics[width=#2\linewidth]{#1}
        \caption{#3} \label{#4}
    \end{figure}
}

\newcommand{\addimghere}[4]{ % Добавить рисунок непосредственно в это место
    \begin{figure}[H]
        \centering
        \includegraphics[width=#2\linewidth]{#1}
        \caption{#3} \label{#4}
    \end{figure}
}

\newcommand{\addtwoimghere}[4]{
	\begin{figure}[h]
		\begin{minipage}[h]{0.5\linewidth}
			\center{\includegraphics[width=0.9\linewidth]{#1} \\ а)}
		\end{minipage}
		\hfill
		\begin{minipage}[h]{0.5\linewidth}
			\center{\includegraphics[width=0.9\linewidth]{#2} \\ б)}
		\end{minipage}
		\caption{#3}
		\label{#4}
	\end{figure}
}

\newcommand{\addtwoimgherepro}[6]{
	\begin{figure}[h]
		\begin{minipage}[h]{#5\linewidth}
			\center{\includegraphics[width=1\linewidth]{#1} \\ а)}
		\end{minipage}
		\hfill
		\begin{minipage}[h]{#6\linewidth}
			\center{\includegraphics[width=1\linewidth]{#2} \\ б)}
		\end{minipage}
		\caption{#3}
		\label{#4}
	\end{figure}
}

\newcommand{\addimgsandhistshere}[6]{
	\begin{figure}[hp]
		\begin{minipage}[hp]{0.3\linewidth}
			\center{\includegraphics[width=1\linewidth]{#1}} \\a) 
		\end{minipage}
		\begin{minipage}[hp]{0.7\linewidth}
			\center{\includegraphics[width=1\linewidth]{#2}} \\б)
		\end{minipage}
		\vfill
		\begin{minipage}[hp]{0.3\linewidth}
			\center{\includegraphics[width=1\linewidth]{#3}} \\в)
		\end{minipage}
		\begin{minipage}[hp]{0.7\linewidth}
			\center{\includegraphics[width=1\linewidth]{#4}} \\г)
		\end{minipage}
		\caption{#5}
		\label{#6}
	\end{figure}
}

\newcommand{\addtwoxtwoimghere}[6]{
	\begin{figure}[hp]
		\begin{minipage}[hp]{0.5\linewidth}
			\center{\includegraphics[width=0.7\linewidth]{#1}} \\a) 
		\end{minipage}
		\begin{minipage}[hp]{0.5\linewidth}
			\center{\includegraphics[width=0.7\linewidth]{#2}} \\б)
		\end{minipage}
		\vfill
		\begin{minipage}[hp]{0.5\linewidth}
			\center{\includegraphics[width=0.7\linewidth]{#3}} \\в)
		\end{minipage}
		\begin{minipage}[hp]{0.5\linewidth}
			\center{\includegraphics[width=0.7\linewidth]{#4}} \\г)
		\end{minipage}
		\caption{#5}
		\label{#6}
	\end{figure}
}



%
%% Заголовки секций в оглавлении в верхнем регистре
%\usepackage{textcase}
%\makeatletter
%\let\oldcontentsline\contentsline
%\def\contentsline#1#2{
%    \expandafter\ifx\csname l@#1\endcsname\l@section
%        \expandafter\@firstoftwo
%    \else
%        \expandafter\@secondoftwo
%    \fi
%    {\oldcontentsline{#1}{\MakeTextUppercase{#2}}}
%    {\oldcontentsline{#1}{#2}}
%}
%\makeatother
%
%% Оформление заголовков
%\usepackage[compact,explicit]{titlesec}
%\titleformat{\section}{}{}{12.5mm}{\centering{\thesection\quad\MakeTextUppercase{#1}}\vspace{1.5em}}
%\titleformat{\subsection}[block]{\vspace{1em}}{}{12.5mm}{\thesubsection\quad#1\vspace{1em}}
%\titleformat{\subsubsection}[block]{\vspace{1em}\normalsize}{}{12.5mm}{\thesubsubsection\quad#1\vspace{1em}}
%\titleformat{\paragraph}[block]{\normalsize}{}{12.5mm}{\MakeTextUppercase{#1}}
%
%% Секции без номеров (введение, заключение...), вместо section*{}
\newcommand{\anonsection}[1]{
    \phantomsection % Корректный переход по ссылкам в содержании
    \paragraph{\LARGE{\leftline{{#1}}}\vspace{1em}}
    \addcontentsline{toc}{section}{#1}
}

%
%% Секции для приложений
%\newcommand{\appsection}[1]{
%    \phantomsection
%    \paragraph{\centerline{{#1}}}
%    \addcontentsline{toc}{section}{\uppercase{#1}}
%}
%
%% Библиография: отступы и межстрочный интервал
%\makeatletter
%\renewenvironment{thebibliography}[1]
%    {\section*{\refname}
%        \list{\@biblabel{\@arabic\c@enumiv}}
%           {\settowidth\labelwidth{\@biblabel{#1}}
%            \leftmargin\labelsep
%            \itemindent 16.7mm
%            \@openbib@code
%            \usecounter{enumiv}
%            \let\p@enumiv\@empty
%            \renewcommand\theenumiv{\@arabic\c@enumiv}
%        }
%        \setlength{\itemsep}{0pt}
%    }
%\makeatother

\usepackage{listings}
\usepackage{color}
\usepackage{minted}
\usepackage{python}
%\definecolor{dkgreen}{rgb}{0,0.6,0}
%\definecolor{gray}{rgb}{0.5,0.5,0.5}
%\definecolor{mauve}{rgb}{0.58,0,0.82}
%
\lstset{ %
language=Python,                 % выбор языка для подсветки (здесь это С)
basicstyle=\small\sffamily, % размер и начертание шрифта для подсветки кода
numbers=left,               % где поставить нумерацию строк (слева\справа)
numberstyle=\tiny,           % размер шрифта для номеров строк
stepnumber=1,                   % размер шага между двумя номерами строк
numbersep=5pt,                % как далеко отстоят номера строк от подсвечиваемого кода
backgroundcolor=\color{white}, % цвет фона подсветки - используем \usepackage{color}
showspaces=false,            % показывать или нет пробелы специальными отступами
showstringspaces=false,      % показывать или нет пробелы в строках
showtabs=false,             % показывать или нет табуляцию в строках
frame=single,              % рисовать рамку вокруг кода
tabsize=2,                 % размер табуляции по умолчанию равен 2 пробелам
captionpos=t,              % позиция заголовка вверху [t] или внизу [b] 
breaklines=true,           % автоматически переносить строки (да\нет)
breakatwhitespace=false, % переносить строки только если есть пробел
escapeinside={\%*}{*)}   % если нужно добавить комментарии в коде
}

 
\titleformat{\paragraph}[display]
    {\filcenter}
    {\MakeUppercase{\chaptertitlename} \thechapter}
    {8pt}
    {\bfseries}{}
\titlespacing*{\paragraph}{0pt}{-30pt}{8pt}
 
\newcommand{\append}[1]{  
    \clearpage
    \stepcounter{chapter}    
    \paragraph{\MakeUppercase{#1}}
    \empline
    \addcontentsline{toc}{likechapter}{\MakeUppercase{\chaptertitlename~\Asbuk{chapter}\;#1}}}

%\setcounter{page}{4} % Начало нумерации страниц
\usepackage [utf8]{inputenc}

% \sloppy
\fussy
\makeatletter

\begin{document}
ТИТУЛЬНИК
\bigskip
\noindent Однажды в студеную зимнюю пору, \\
Сижу за решеткой в темнице сырой.\\
Гляжу, поднимается медленно в гору\\
Вскормленный в неволе орел молодоц,\\
И, шествую важно, в спокойствии чинном\\
Мой грустный товарищ, махая крылом,\\
В больших сапогах, в полушибке овчинном\\
Кровавую пищу клюет под окном.
\bigskip

\tableofcontents
\section{First section} \label{first_section}
Hello World!
\subsection{First subsection}
\section*{Not numbered section}
\section{Second section}
The subject discussed in Section \ref{first_section} on \pageref{first_section} is very interesting. Let me tell you why...
\newcommand {\E}[0]{$E_8^{+++}$}
\newcommand {\seclab}[1]{\section{Section: #1}}
\seclab {Something}
\seclab {Introduction}
\let\oldsection\section
\renewcommand {\section}[1]{\centerline{Section: #1}}

\def \E {$E_8^{+++}$}
\E

\newenvironment{ruled}[0]{before\hrule}{after\hrule}


\begin{ruled}
some text	
\end{ruled}

\oldsection {TEsting}

У попа {\it была} {\bf собака}

В этой \itshape фразе почти все \bfseries
набрано курсивом, \ttfamily но кое-что
\upshape ещё и жирное, \rmfamily а где-то
и вовсе \mdseries моноширинное.
Тут сначала задали курсив, потом жирное, затем моноширинное (жирность перестала работать), 
потом отменили курсив и вернули к обычной, затем последним словом сняли жирность.
\normalfont

\textbackslash sffamily - {\sffamily гарнитура без засечек}

\textbackslash ttfamily - {\ttfamily моноширинная гарнитура }

\textbackslash rmfamily - {\rmfamily обычная гарнитура }

\textbackslash bfseries - {\bfseries включение жирности }

\textbackslash mdseries - {\mdseries отмена жирности }

\textbackslash itshape - {\itshape курсив }

\textbackslash slshape - {\slshape наклонный шрифт }

\textbackslash scshape - {\scshape капитель }

\textbackslash upshape - {\upshape обычное начертание }

Команды \ttfamily {
\textbackslash textbf\{\}, 
\textbackslash texttt\{\},
\textbackslash textsf\{\}, 
\textbackslash textrm\{\},
\textbackslash textmd\{\},
\textbackslash textit\{\},
\textbackslash textsl\{\},
\textbackslash textsc\{\},
\textbackslash textup\{\},
\textbackslash textnormal\{\},
\textbackslash emph\{\}
}\normalfont

Пример \texttt{\textbackslash emph\{\}}: \emph{ВыДеЛеНиЕ}
  
\subsubsection{Пример с окружением}

Сначала\begin{bfseries} жирное, потом\begin{itshape} курсив, потом\end{itshape}
без курсива, потом\end{bfseries} без жирного.

\subsubsection{Устаревшая форма команд}

Команды \ttfamily {
\textbackslash bf, \textbackslash it, \textbackslash sl, \textbackslash sc,
\textbackslash sf, \textbackslash tt, \textbackslash rm
}

\bf Их эффект не может быть наложен! \label{label1}
\normalfont
\subsection{Размер шрифта} 
\subsubsection{Размеры шрифта и команды}

\ttfamily{\textbackslash tiny} - \begin{tiny} tiny \end{tiny}

\ttfamily{\textbackslash scriptsize} - \begin{scriptsize} scriptsize \end{scriptsize}

\ttfamily{\textbackslash footnotesize} - \begin{footnotesize} footnotesize \end{footnotesize}

\ttfamily{\textbackslash small} - \begin{small} small \end{small}

\ttfamily{\textbackslash normalsize} - \begin{normalsize} normalsize \end{normalsize}

\ttfamily{\textbackslash large} - \begin{large} large \end{large}

\ttfamily{\textbackslash Large} - \begin{Large} Large \end{Large}

\ttfamily{\textbackslash LARGE} - \begin{LARGE} LARGE \end{LARGE}

\ttfamily{\textbackslash huge} - \begin{huge} huge \end{huge}

\ttfamily{\textbackslash Huge} - \begin{Huge} Huge \end{Huge}

\normalfont

\subsection{Рубрикация}

\chapter{Введение}


Как нам начать писать курсач? Наверное просто стоит начать хотя бы что-то писать, а то какая-то дичь вообще получается.

\oldsection{Постановка задачи}

\oldsection{ПУНКТЫ}

\begin{itemize}
	\item 1
	\item 2
	\item 3
	\begin{itemize}
	\item 3.1
	\end{itemize}
\end{itemize}

\begin{enumerate}
	\item 1
	\item 2
	\item 3
	\begin{enumerate}
		\item 3.1
		\item 3.2
	\end{enumerate}
\end{enumerate}

\begin{compactlist}
	\item 1
	\item 2
	\item 3
	\item 4
\end{compactlist}



ТЕКСТ ТЕКСТ \cite{kniga1}

ТЕКСТ ТЕКСТ \cite{kniga2}

\oldsection{ТАБЛИЦЫ И РИСУНКИ}



\begin{table}[t]
	\centering
	\begin{tabular}
		{|r|c|c|c|c|r|}
		\hline No & Book & Author & Column1 & Column2 & Тираж \\ \hline 
		1 & Книга &  F & \multicolumn{2}{c|}{FFFF 1111} & 10\,000 \\ \hline 
	\end{tabular}
	\caption{Название таблицы}
	\label{table1}
\end{table}

\begin{figure}[h]
	\centering
	\includegraphics[width=1\textwidth]{testimage}
	\caption{Название рисунка}
	\label{ris1}
\end{figure}

Смари рис. \ref{ris1}!!!!

\oldsection{ЛИСТИНГИ ПРОГРАММ}


\begin{verbatim}
	program example;
	begin
	{ This program just prints a message }
		writeln('Hello world');
	end.
\end{verbatim}

\begin{listing}[3]{1}
	program example;
	begin
	{ This program just prints a message }
		writeln('Hello world');
	end.
	end.
\end{listing}

\oldsection{СНОСКИ И ЗАМЕТКИ}

Текст\footnote{сноска \label{my_footnote}} текст

%Текст \footnotemark[\ref{my_footnote}].

\marginpar[\hfill\attentionpicture]{\attentionpicture}

\oldsection{МАТЕМАТИЧЬКА}

Формула \begin{math} x^2 + y^2 = 0 \end{math} 
\begin{displaymath}
	x^2 + y^2 = 0	
\end{displaymath}

Формула \( x^2 + y^2 = 0 \) 
\[ x^2 + y^2 = 0\]

Формула $ x^2 + y^2 = 0$ $$ x^2 + y^2 = 0 $$

Если функция \( F(x) \) является одной из первообразных функции \( f(x) \) на интервале
\( (a, b) \), то
\[\int f(x)\,dx=F(x)+C,\]
где \(C\)~--- произвольная постоянная

Пробелы
\[ x\negthickspace x \]
\[ x\negmedspace x \]
\[ x\!x \]
\[ xx \]
\[ x\,x \]
\[ x\:x \]
\[ x\quad x \]
\[ x\qquad x \]

Формулы с текстом
\[ V_\text{сближения} = V_\text{автомобиля} + V_\text{велосипедиста} \]

Дроби

\( \frac{x+\frac{1}{y}}{\frac{z+1}{3}-15} \)
\[ \frac{x+\frac{1}{y}}{\frac{z+1}{3}-15} \]

\[ \sqrt{1 - \sin^2 x} + \sqrt[4]{1 + \cos^2 x} = 0 \]

\[ \sqrt{x_1 + \sqrt{x_2 + \sqrt{x_3 + \sqrt {x_4}}}} \]

Интеграл 
\[ \int_a^b f(x)\,dx \]
\[ \int\limits_a^b f(x)\,dx \]
\[ \iiint\limits_Vf(v) f(v),dv \]
"Круглый" интеграл 
\[ \oint_a^b f(x)\,dx \]
\[ \oint\limits_a^b f(x)\,dx \]
Сумма \[ \sum_{i=0}^{n} q_i \] \[ \sum\limits_{i=0}^{n} q_i \]

Символы всякие полезные
\newcommand\itemmathsign[1]{\ttfamily{\textbackslash #1}}
\begin{compactlist}
	\item \(\forall\) - \itemmathsign{forall}
	\item $\in$ - \itemmathsign{in}
	\item $\subset$ - \itemmathsign{subset}
	\item $\leq$ - \itemmathsign{leq}
	\item $\cdot$ - \itemmathsign{cdot}
	\item $\ll$ - \itemmathsign{ll}
	\item $\vee$ - \itemmathsign{vee}
	\item $\perp$ - \itemmathsign{perp}
	\item $\lfloor$ - \itemmathsign{lfloor}
	\item $\pm$ - \itemmathsign{pm}
	\item $\oplus$ - \itemmathsign{oplus}
	\item $\odot$ - \itemmathsign{odot}
	\item $\exists$ - \itemmathsign{exists}
	\item $\ni$ - \itemmathsign{ni}
	\item $\supset$ - \itemmathsign{supset}
	\item $\geq$ - \itemmathsign{geq}
	\item $\leqslant$ - \itemmathsign{leqslant}
	\item $\geqslant$ - \itemmathsign{geqslant}
	\item $\approx$ - \itemmathsign{approx}
	\item $\gg$ - \itemmathsign{gg}
	\item $\infty$ - \itemmathsign{infty}
	\item $\neq$ - \itemmathsign{neq}
	\item $\times$ - \itemmathsign{times}
	\item $\circ$ - \itemmathsign{circ}
	\item $\leftarrow$ - \itemmathsign{leftarrow}
	\item $\Leftarrow$ - \itemmathsign{Leftarrow}
	\item $\leftrightarrow$ - \itemmathsign{leftrigtharrow}
	\item $\longleftarrow$ - \itemmathsign{longleftarrow}
	\item $\Longleftarrow$ - \itemmathsign{Longleftarrow}
	\item $\longleftrightarrow$ - \itemmathsign{longleftrightarrow}
	\item $\Longleftrightarrow$ - \itemmathsign{Longleftrightarrow}
	\item $\Leftrightarrow$ - \itemmathsign{Leftrightarrow}
\end{compactlist}

\[ \forall\varepsilon>0 \quad \exists\delta(\varepsilon)>0\;:\;\forall x 
\; 0<|x-a|<\delta \Rightarrow |f(x)-b|<\varepsilon \]

\begin{compactlist}
	\item $\hat x$ - \itemmathsign{hat}
	\item $\dot x$ - \itemmathsign{dot}
	\item $\bar x$ - \itemmathsign{bar}
	\item $\mathring x$ - \itemmathsign{mathring}
	\item $\check x$ - \itemmathsign{check}
	\item $\ddot x$ - \itemmathsign{ddot}
	\item $\vec x$ - \itemmathsign{vec}
	\item $\acute x$ - \itemmathsign{acute}
	\item $\dddot x$ - \itemmathsign{dddot}
	\item $\breve x$ - \itemmathsign{breve}
	\item $\grave x$ - \itemmathsign{grave}
	\item $\ddddot x$ - \itemmathsign{ddddot}
	\item $\tilde x$ - \itemmathsign{tilde}
	\item $1, 2, \ldots, n$ - \itemmathsign{ldots}
	\item $1, 2, \cdots, n$ - \itemmathsign{cdots}
	\item $\vdots$ - \itemmathsign{vdots}
	\item $\ddots$ - \itemmathsign{ddots}
	\item $(x+y)$ - \ttfamily{(x+y)}
	\item $[x+y]$ - \ttfamily{[x+y]}
	\item $\{x+y\}$ - \ttfamily{\textbackslash \{x + y\textbackslash \}}
	\item $\langle x + y \rangle$ - \itemmathsign{langle} x + y \itemmathsign{rangle}
	\item $\lfloor x + y \rfloor$ - \itemmathsign{lfloor} x + y \itemmathsign{rfloor}
	\item $\lceil x + y \rceil$ - \itemmathsign{lceil} x + y \itemmathsign{lceil}
\end{compactlist}

\oldsection{Стили матем формул}

\[1+\frac{1}{1+\frac{1}{1+\frac{1}{1 + \frac{1}{1+\frac{1}{x}}}}}\]
\[1+\frac{1}{\displaystyle1+\frac{1}
{\displaystyle1+\frac{1}
{\displaystyle1 + \frac{1}
{\displaystyle1+\frac{1}{x}}}}}\]

\oldsection{Матрицы}

\[ \left( \begin{matrix} 1 & 2 & 3 \\ x & y & z \\ a & b & c \\ \end{matrix} \right) \]

\[
\begin{pmatrix} a_1^1 & a_1^2 \\ a_2^1 & a_2^2 \end{pmatrix}
\begin{bmatrix} a_1^1 & a_1^2 \\ a_2^1 & a_2^2 \end{bmatrix}
\begin{vmatrix} a_1^1 & a_1^2 \\ a_2^1 & a_2^2 \end{vmatrix}
\begin{Vmatrix} a_1^1 & a_1^2 \\ a_2^1 & a_2^2 \end{Vmatrix}
\]

% по умолчанию максимум -- 10 столбцов!
% для расширения можно \setcounter{MaxMatrixCols}{20} 
% но потом отменить, потому что тратится больше времени на вёрстку

Маленькая матрица в строчке \(
\left( \begin{smallmatrix}
	a & b \\ c & d
\end{smallmatrix}\right) 
\)

\oldsection{Системы уравнений}

\subsection{Выравнивание формул}
Тут выравнивание по знаку равенства:
\[\begin{aligned}
\sin x\pm\sin y &= 2\sin\frac{x\pm y}{2}\cos\frac{x\mp y}{2} \\
\sin x\sin y &= \frac{1}{2}(\cos(x-y)-\cos(x+y)) \\
\end{aligned}\]

\subsection{Системы уравнений и неравенств}

\[ \left\{\begin{aligned}
	& x+y=5 \\
	& x-2y=2
\end{aligned}\right. \]

\[ 
	\sqrt{f(x)} > g(x) \Longleftrightarrow 
	\left[ 
	\begin{aligned}
		& \left\{ 
			\begin{aligned}
				& f(x) > (g(x))^2 \\
				& g(x) \geq 0
			\end{aligned}
		  \right . \\
		& \left\{
			\begin{aligned}
				& f(x) \geq 0 \\
				& g(x) < 0
			\end{aligned} 
		  \right . \\
	\end{aligned} \right.
\]

\subsection{Перечисление случаев}

\[ |x| = \begin{cases}
 		x, &x \ge 0, \\
 		-x, &x < 0	
 \end{cases} \]
 
 \oldsection{Играемся}
 
 \begin{centering}
 
Текст по центру
 
 \end{centering}
 
 А теперь нет
 
 \hrulefill
 
\null\hfill
\begin{minipage}{0.3\textwidth}
 	Сдвиг вправо и текст текст ектстект сткттадылва офжвафжывадфы вдафдвоадфодывоад фодывод афо дывода жфдывалоф
\end{minipage}\\
 
\null\hfill
\begin{boxedminipage}{0.3\textwidth}
 	Сдвиг вправо и бордюр текст текстект сте ктстктта дылва офж вафжы вадфывдафдвоа дфодыв оадфо дыводаф одыво дажфд
\end{boxedminipage}

\subsection{Пример титульника}
\clearpage
\thispagestyle{empty}
\centerline{ИНСТИТУТ БРЕВЕН И СУЧКОВ РАН}
\centerline{\hfill\hrulefill\hrulefill\hfill}
\vfill
\rightline{на правах рукописи}
\vfill
\vfill
\large
\centerline{{\bf Юдаков} Дмитрий Игоревич}
\vfill
\Large
\begin{centering}
	{\bf Качение бревна\\ по наклонной плоскости\\
	с учётом сучковатости\\}
\end{centering}
\normalsize
\vfill
\centerline{Специальность 66.69.99~--- }
\centerline{механические и кинематические свойства}
\centerline{сучковатых бревен}
\vfill
\centerline{Диссертация на соискание учёной степени}
\centerline{кандидата бревнологических наук}
\vfill
\vfill
\begin{flushright}
Научный руководитель:\\
д.\,бр.\,н.~Персикович~И.\,И.
\end{flushright}
\vfill
\vfill
\centerline{Москва~--- 2021}

\tableofcontents


ddddd \ref{label1}

\appendix


\chapter{xxx}

xxxxjjjj

\renewcommand\bibname{Cписок литературы}
\begin{thebibliography}{00}

	\bibitem{kniga1} ДЖЫЛВАОЫЛВАЫЖЛВАЫД
	
	\bibitem{kniga2} ЛДРВАРЫРВРАРЫДВА 
	
\end{thebibliography}

\end{document}















%%%%% Преамбула %%%
%
%\usepackage{fontspec} % XeTeX
%\usepackage{xunicode} % Unicode для XeTeX
%\usepackage{xltxtra}  % Верхние и нижние индексы
%\usepackage{pdfpages} % Вставка PDF
%
\usepackage[usenames,dvipsnames]{color}
\usepackage{geometry}           % пакет для задания полей страницы командой \geometry
\geometry{left=3cm,right=1.5cm,top=2cm,bottom=2cm}
\usepackage[utf8]{inputenc}   % кодировка текста
\usepackage{mathtext}           % позволяет использовать русские буквы в формулах
\usepackage[T2A]{fontenc}       %пакет Т2А необходим для правильного отображения кириллицы и переноса слов
%\inputencoding{cp1251}          % тоже кодировка...
\usepackage[russian]{babel}     % языковой пакет - последний язык главный
\usepackage[unicode]{hyperref}  %создаёт гиперссылки на список литературы в pdf-файле
\usepackage{amstext,amsmath,amssymb}            % пакеты для формул
\usepackage{bm}                 % boldmath - пакет для жирного шрифта
\usepackage[pdftex]{graphicx}   % пакет для включения рисунков в форматах png,pdf,jpg,mps,tif
\usepackage{titlesec}
\usepackage[title,titletoc]{appendix}

\usepackage{amsfonts}           % греческие символы и, возможно, что-то ещё
\usepackage{indentfirst}        % одинаковый отступ для первого параграфа и всего остального
\usepackage{cite}               % команда /cite{1,2,7,9} даёт ссылки
\usepackage{multirow}           % пакет для объединения строк в таблице: надо указать число строк и ширину столбца
\usepackage{array}              % нужен для создания таблиц
\linespread{1.3}                % полтора интервала. Если 1.6, то два интервала
\pagestyle{plain}               % номерует страницы

\usepackage[title,titletoc]{appendix}
\usepackage{listings} % Оформление исходного кода
\lstset{
	language=Python,
    basicstyle=\small\ttfamily, % Размер и тип шрифта
    breaklines=true, % Перенос строк
    tabsize=2, % Размер табуляции
    literate={--}{{-{}-}}2 % Корректно отображать двойной дефис
}
%
%% Шрифты, xelatex
%\defaultfontfeatures{Ligatures=TeX}
%\setmainfont{Times New Roman} % Нормоконтроллеры хотят именно его
%\newfontfamily\cyrillicfont{Times New Roman}
%%\setsansfont{Liberation Sans} % Тут я его не использую, но если пригодится
%\setmonofont{FreeMono} % Моноширинный шрифт для оформления кода
%
%% Русский язык
%\usepackage{polyglossia}
%\usepackage{amssymb,amsfonts,amsmath} % Математика
%\numberwithin{equation}{section} % Формула вида секция.номер
%
%\usepackage{enumerate} % Тонкая настройка списков
%\usepackage{indentfirst} % Красная строка после заголовка
%\usepackage{float} % Расширенное управление плавающими объектами
%\usepackage{multirow} % Сложные таблицы
%
%% Пути к каталогам с изображениями
%\usepackage{graphicx} % Вставка картинок и дополнений
%\graphicspath{{images/}{images/userguide/}{images/testing/}{images/infrastructure/}{extra/}{extra/drafts/}}
%
%% Формат подрисуночных записей
%\usepackage{chngcntr}
%\counterwithin{figure}{section}
%
%% Гиперссылки
\usepackage{hyperref}
\hypersetup{
    colorlinks, urlcolor={black}, % Все ссылки черного цвета, кликабельные
    linkcolor={black}, citecolor={black}, filecolor={black},
    pdfauthor={Дмитрий Юдаков},
    pdftitle={Исследование методов и разработка алгоритма детектирования, описания и извлечения конфигураций кисти человека}
}
%
%% Оформление библиографии и подрисуночных записей через точку
%\makeatletter
%\renewcommand*{\@biblabel}[1]{\hfill#1.}
%\renewcommand*\l@section{\@dottedtocline{1}{1em}{1em}}
%\renewcommand{\thefigure}{\thesection.\arabic{figure}} % Формат рисунка секция.номер
%\renewcommand{\thetable}{\thesection.\arabic{table}} % Формат таблицы секция.номер
%\def\redeflsection{\def\l@section{\@dottedtocline{1}{0em}{10em}}}
%\makeatother
%
%\renewcommand{\baselinestretch}{1.4} % Полуторный межстрочный интервал
%\parindent 1.27cm % Абзацный отступ
%
%\sloppy             % Избавляемся от переполнений
%\hyphenpenalty=1000 % Частота переносов
%\clubpenalty=10000  % Запрещаем разрыв страницы после первой строки абзаца
%\widowpenalty=10000 % Запрещаем разрыв страницы после последней строки абзаца
%
%% Отступы у страниц
%\usepackage{geometry}
%\geometry{left=3cm}
%\geometry{right=1cm}
%\geometry{top=2cm}
%\geometry{bottom=2cm}
%
%% Списки
\usepackage{enumitem}
\setlist[enumerate,itemize]{leftmargin=12.7mm} % Отступы в списках
%
%\makeatletter
%    \AddEnumerateCounter{\asbuk}{\@asbuk}{м)}
%\makeatother
\setlist{nolistsep} % Нет отступов между пунктами списка
%\renewcommand{\labelitemi}{--} % Маркет списка --
%\renewcommand{\labelenumi}{\asbuk{enumi})} % Список второго уровня
%\renewcommand{\labelenumii}{\arabic{enumii})} % Список третьего уровня
%
%% Содержание
%\usepackage{tocloft}
%\renewcommand{\cfttoctitlefont}{\hspace{0.38\textwidth}\MakeTextUppercase} % СОДЕРЖАНИЕ
%\renewcommand{\cftsecfont}{\hspace{0pt}}            % Имена секций в содержании не жирным шрифтом
%\renewcommand\cftsecleader{\cftdotfill{\cftdotsep}} % Точки для секций в содержании
%\renewcommand\cftsecpagefont{\mdseries}             % Номера страниц не жирные
%\setcounter{tocdepth}{3}                            % Глубина оглавления, до subsubsection
%
%%% Нумерация страниц справа сверху
%%\usepackage{fancyhdr}
%%\pagestyle{fancy}
%%\fancyhf{}
%%\fancyhead[R]{\textrm{\thepage}}
%%\fancyheadoffset{0mm}
%%\fancyfootoffset{0mm}
%%\setlength{\headheight}{17pt}
%%\renewcommand{\headrulewidth}{0pt}
%%\renewcommand{\footrulewidth}{0pt}
%%\fancypagestyle{plain}{ 
%%    \fancyhf{}
%%    \rhead{\thepage}
%%}
%
%% Формат подрисуночных надписей
\RequirePackage{caption}
\DeclareCaptionLabelSeparator{defffis}{. } % Разделитель
\captionsetup[figure]{justification=centering, labelsep=defffis, format=plain} % Подпись рисунка по центру
\captionsetup[table]{justification=centering, labelsep=defffis, format=plain, singlelinecheck=false} % Подпись таблицы слева
\addto\captionsrussian{\renewcommand{\figurename}{Рис.}} % Имя фигуры

%% Пользовательские функции


%\newcommand{\addimgapp}[2]{ % Это костыль для приложения Б
%    \begin{figure}[H]
%        \centering
%        \includegraphics[width=1\linewidth]{#1}
%        \caption*{#2}
%    \end{figure}
%}


\newcommand{\addimg}[4]{ % Добавление одного рисунка
    \begin{figure}
        \centering
        \includegraphics[width=#2\linewidth]{#1}
        \caption{#3} \label{#4}
    \end{figure}
}

\newcommand{\addimghere}[4]{ % Добавить рисунок непосредственно в это место
    \begin{figure}[H]
        \centering
        \includegraphics[width=#2\linewidth]{#1}
        \caption{#3} \label{#4}
    \end{figure}
}

\newcommand{\addtwoimghere}[4]{
	\begin{figure}[h]
		\begin{minipage}[h]{0.5\linewidth}
			\center{\includegraphics[width=0.9\linewidth]{#1} \\ а)}
		\end{minipage}
		\hfill
		\begin{minipage}[h]{0.5\linewidth}
			\center{\includegraphics[width=0.9\linewidth]{#2} \\ б)}
		\end{minipage}
		\caption{#3}
		\label{#4}
	\end{figure}
}

\newcommand{\addtwoimgherepro}[6]{
	\begin{figure}[h]
		\begin{minipage}[h]{#5\linewidth}
			\center{\includegraphics[width=1\linewidth]{#1} \\ а)}
		\end{minipage}
		\hfill
		\begin{minipage}[h]{#6\linewidth}
			\center{\includegraphics[width=1\linewidth]{#2} \\ б)}
		\end{minipage}
		\caption{#3}
		\label{#4}
	\end{figure}
}

\newcommand{\addimgsandhistshere}[6]{
	\begin{figure}[hp]
		\begin{minipage}[hp]{0.3\linewidth}
			\center{\includegraphics[width=1\linewidth]{#1}} \\a) 
		\end{minipage}
		\begin{minipage}[hp]{0.7\linewidth}
			\center{\includegraphics[width=1\linewidth]{#2}} \\б)
		\end{minipage}
		\vfill
		\begin{minipage}[hp]{0.3\linewidth}
			\center{\includegraphics[width=1\linewidth]{#3}} \\в)
		\end{minipage}
		\begin{minipage}[hp]{0.7\linewidth}
			\center{\includegraphics[width=1\linewidth]{#4}} \\г)
		\end{minipage}
		\caption{#5}
		\label{#6}
	\end{figure}
}

\newcommand{\addtwoxtwoimghere}[6]{
	\begin{figure}[hp]
		\begin{minipage}[hp]{0.5\linewidth}
			\center{\includegraphics[width=0.7\linewidth]{#1}} \\a) 
		\end{minipage}
		\begin{minipage}[hp]{0.5\linewidth}
			\center{\includegraphics[width=0.7\linewidth]{#2}} \\б)
		\end{minipage}
		\vfill
		\begin{minipage}[hp]{0.5\linewidth}
			\center{\includegraphics[width=0.7\linewidth]{#3}} \\в)
		\end{minipage}
		\begin{minipage}[hp]{0.5\linewidth}
			\center{\includegraphics[width=0.7\linewidth]{#4}} \\г)
		\end{minipage}
		\caption{#5}
		\label{#6}
	\end{figure}
}



%
%% Заголовки секций в оглавлении в верхнем регистре
%\usepackage{textcase}
%\makeatletter
%\let\oldcontentsline\contentsline
%\def\contentsline#1#2{
%    \expandafter\ifx\csname l@#1\endcsname\l@section
%        \expandafter\@firstoftwo
%    \else
%        \expandafter\@secondoftwo
%    \fi
%    {\oldcontentsline{#1}{\MakeTextUppercase{#2}}}
%    {\oldcontentsline{#1}{#2}}
%}
%\makeatother
%
%% Оформление заголовков
%\usepackage[compact,explicit]{titlesec}
%\titleformat{\section}{}{}{12.5mm}{\centering{\thesection\quad\MakeTextUppercase{#1}}\vspace{1.5em}}
%\titleformat{\subsection}[block]{\vspace{1em}}{}{12.5mm}{\thesubsection\quad#1\vspace{1em}}
%\titleformat{\subsubsection}[block]{\vspace{1em}\normalsize}{}{12.5mm}{\thesubsubsection\quad#1\vspace{1em}}
%\titleformat{\paragraph}[block]{\normalsize}{}{12.5mm}{\MakeTextUppercase{#1}}
%
%% Секции без номеров (введение, заключение...), вместо section*{}
\newcommand{\anonsection}[1]{
    \phantomsection % Корректный переход по ссылкам в содержании
    \paragraph{\LARGE{\leftline{{#1}}}\vspace{1em}}
    \addcontentsline{toc}{section}{#1}
}

%
%% Секции для приложений
%\newcommand{\appsection}[1]{
%    \phantomsection
%    \paragraph{\centerline{{#1}}}
%    \addcontentsline{toc}{section}{\uppercase{#1}}
%}
%
%% Библиография: отступы и межстрочный интервал
%\makeatletter
%\renewenvironment{thebibliography}[1]
%    {\section*{\refname}
%        \list{\@biblabel{\@arabic\c@enumiv}}
%           {\settowidth\labelwidth{\@biblabel{#1}}
%            \leftmargin\labelsep
%            \itemindent 16.7mm
%            \@openbib@code
%            \usecounter{enumiv}
%            \let\p@enumiv\@empty
%            \renewcommand\theenumiv{\@arabic\c@enumiv}
%        }
%        \setlength{\itemsep}{0pt}
%    }
%\makeatother

\usepackage{listings}
\usepackage{color}
\usepackage{minted}
\usepackage{python}
%\definecolor{dkgreen}{rgb}{0,0.6,0}
%\definecolor{gray}{rgb}{0.5,0.5,0.5}
%\definecolor{mauve}{rgb}{0.58,0,0.82}
%
\lstset{ %
language=Python,                 % выбор языка для подсветки (здесь это С)
basicstyle=\small\sffamily, % размер и начертание шрифта для подсветки кода
numbers=left,               % где поставить нумерацию строк (слева\справа)
numberstyle=\tiny,           % размер шрифта для номеров строк
stepnumber=1,                   % размер шага между двумя номерами строк
numbersep=5pt,                % как далеко отстоят номера строк от подсвечиваемого кода
backgroundcolor=\color{white}, % цвет фона подсветки - используем \usepackage{color}
showspaces=false,            % показывать или нет пробелы специальными отступами
showstringspaces=false,      % показывать или нет пробелы в строках
showtabs=false,             % показывать или нет табуляцию в строках
frame=single,              % рисовать рамку вокруг кода
tabsize=2,                 % размер табуляции по умолчанию равен 2 пробелам
captionpos=t,              % позиция заголовка вверху [t] или внизу [b] 
breaklines=true,           % автоматически переносить строки (да\нет)
breakatwhitespace=false, % переносить строки только если есть пробел
escapeinside={\%*}{*)}   % если нужно добавить комментарии в коде
}

 
\titleformat{\paragraph}[display]
    {\filcenter}
    {\MakeUppercase{\chaptertitlename} \thechapter}
    {8pt}
    {\bfseries}{}
\titlespacing*{\paragraph}{0pt}{-30pt}{8pt}
 
\newcommand{\append}[1]{  
    \clearpage
    \stepcounter{chapter}    
    \paragraph{\MakeUppercase{#1}}
    \empline
    \addcontentsline{toc}{likechapter}{\MakeUppercase{\chaptertitlename~\Asbuk{chapter}\;#1}}}

%\setcounter{page}{4} % Начало нумерации страниц