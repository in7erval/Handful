\anonsection{ВВЕДЕНИЕ}

Существуют ситуации, в которых необходимо взаимодействовать с системой без прикосновений.
Причинами могут служить: грязные руки, при работе инженера с оборудованием, гигиена, 
например, чтобы настроить желаемую температуру воды при мытье рук, и фокусировка внимания,
чтобы не перенаправлять взгляд на элементы управления при работе с хрупким оборудованием
или при взаимодействии со сценариями дополненной реальности. Использование голосовых команд
в качества альтернативы сенсорному управлению, например, клавиатуры, кнопок и сенсорных экранов,
требует тихой среды и обработки естественного языка. Кроме того, голосовые команды
зависят от языка и чувствительны к диалектам и дефектам речи. Другой альтернативой является
дистанционное управление посредством распознавания жестов, также известное как
дистанционное управление "взмахом руки". Обычные применения для этого вида управления 
включают медицинские системы, обеспечивая стерильность пользователя и исключения 
распространения инфекций, развлечения и взаимодействие человека и работа или машины.

В данной работе идёт речь о поиске так называемых "особых"\ точек на кисти руки 
для дальнейшей их обработки. Поиск таких точек позволяет извлекать конфигурацию кисти
и давать её описание, что позволяет не только задавать исполнение определённых 
команд по заранее определённым жестам, но и выполнять совершенно не характерные для обычной
реализации данного метода действия.

В ходе исследования разрабатывается алгоритм детектирования кисти руки человека.
Важным требованием для реализации является минимизация времени
задержки на обработку каждого кадра для комфортного использования в прикладных задачах.

Поставленной {\bf целью} является создание программы для детектирования кисти руки человека с помощью 
языка высокого уровня Python.

Для достижения цели необходимо решить ряд {\bf задач}:
\begin{enumerate}
	\item Сделать обзор теоретического материала по детектированию кисти руки человека.
	\item Найти и описать наиболее популярные методы обнаружения кисти человека и выделить из них самые эффективные.
	\item Провести сравнение нескольких алгоритмов решения данной задачи.
\end{enumerate}