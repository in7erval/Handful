\anonsection{Заключение}

Детектирование рук и распознавание жестов являются одними из
наиболее популярных прикладных задач компьютерного зрения, но
до сих пор в их решении имеются проблемы с получением точного
результата из-за сложности снижения шума. 

Во-первых, для решения задачи детектирования кисти
были рассмотрены различные подходы, а именно детектирование
на основе цветов в разных цветовых моделях (YCbCr и HSV),
разделения пикселей изображения на две группы -- "фон"\ и
"передняя (полезная) часть", и рекурсивные методы
построения модели фона. В результате исследования было получено,
что наиболее эффективным способом в статическом изображении
является детектирование в цветовой модели HSV, а при 
наблюдении в динамике, то есть на видеофрагменте, то 
наилучшие результаты показал метод MoG. 

Во-вторых, для построения контура кисти на изображении были
исследованы два метода: выделение с помощью оператора Кэнни и 
топологический структурный анализ цифрового бинарного
изображения с помощью отслеживания границ. Последний позволяет
находить точные координаты контура, поэтому является наилучшим
среди выбранных.

В дальнейшем есть возможность получения конфигурации
кисти, например, описывая её "ключевыми"\ точками, 
и её применение в большом множестве задач, включая
распознавание жестов или управление какими-либо автоматическими
системами.

